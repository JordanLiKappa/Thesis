\label{sec:SM:intro}
Particle physics is in the heart of our understanding of
the laws of nature. 
The subject is concerned with the fundamental constituents of the
Universe, the \textit{elementary particles}, and the interactions between them, 
the \textit{forces}. 
The Standard Model (SM) embodies the current understanding of
particle physics, providing a unified picture where the forces between the particles
are themselves described by the exchange of particles. It provides a successful description
of all current experimental data and represents on of the triumphs of mordern physics.

The last missing piece of the SM, the Higgs boson has been observed 
by the ATLAS and CMS experimental at the 
Large Hadron Collider~\cite{HIGG-2012-27,CMS-HIG-12-028}. 
The SM is therefore hailed by many as the most successful theory 
ever conceived. Despite its success, the SM is yet not the ultimate  
theory, as many unanswered questions remain. For example, why the SM
has so many free parameters (26) that have to be input by hand; 
what is the particle content of the dark matter; 
what is the origin of the matter-antimatter assymmetry in the universe?

This chapter is structured as follows. 
Section~\ref{sec:SM} introduces the basic concepts of the SM, including the
gauge theory and fundamental forces and the Higgs mechanism. 
Section~\ref{sec:BSM} gives a hint of beyond the SM (BSM) 
theories that address some of the important and unanswered questions. 
Section~\ref{sec:ML} and section~\ref{sec:stats} outline the machine learning algorithms 
and the statistical methods used in this thesis, respectively.
