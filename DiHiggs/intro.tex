This chapter describes the search for Higgs boson pair production in the
\bbtt\ channel, where one Higgs boson decays to a \bquark\ pair 
and the other to a $\tau$-lepton pair. 
As the two $\tau$-leptons decay either leptonically or hadronically, 
the analysis is divided further into two sub-channels depending on
their decay mode, 
the $b\bar{b}\tau_{\text{lep}}^{\pm}\tau_{\text{had}}^{\mp}$ channel 
(for simplicity, referred to as \lephad channel)
where one of the $\tau$-leptons decays leptonically and the other decays
hadronically, 
and the \bbtthh\ channel (or referred to as \hadhad channel) 
where both of the $\tau$-leptons decay hadronically. 
The decay mode of both $\tau$-leptons decay
leptonically is not considered in this analysis 
due to its insignificant contribution. 
In this thesis, the author will present his work 
in the \lephad channel, and will also show the combination results with 
the \hadhad channel.


% one of the $\tau$-leptons decay leptonically and the other decay hadronically. 
The results of this search are interpreted in terms of resonant and 
non-resonant production of the di-Higgs.
For the non-resonant production, upper limits are set on the SM di-Higgs production
cross section, and exclusion limits are set on the Higgs self-coupling $\lambda_{HHH}$.
% The non-resonant search is also interpreted in terms of the coupling modifiers using a
% Higgs Effective Field Theory approach .
For the resonant search, upper limits are set on the resonance production cross section 
as a function of the resonance mass, targeting a generic spin-0 neutral scalar. 
The \lephad and \hadhad
combination results are published in Ref.\cite{dihiggs-conf}.
The $\lambda_{HHH}$ result is combined with other final states of the di-Higgs production
including \bbbb\ and \bbyy. 
The combination result is published in Ref.~\cite{ATLAS-CONF-2021-052}. 

The analysis strategy is as follows:
first, a set of selections are applied on 
the reconstructed physics objects and kinematics variables which
defines the signal regions and the control region, 
as decribed in section~\ref{sec:DiHiggs:selection};
neural networks trained on the signal regions events are then 
used to extract the various signals, and the output of the algorithm is used
as the final discriminant, as described in section~\ref{sec:DiHiggs:MVA};
the systematic uncertainties considered in this analysis is discussed in 
section~\ref{sec:DiHiggs:systematics};
a profile likelihood fit is then performed simultaneously on all \lephad and \hadhad
signal regions and the control region, with all systematics uncertainties served as nuisance parameters.
The statistical combination result of \bbtt, \bbyy and \bbbb is also studied. 
The setup and the results are shown in section~\ref{sec:DiHiggs:results}.

The author's contributions to the analyses presented in this Chapter are as follows.
As one of the main analysers in the \lephad channel, 
the author produced all data and MC (including all signals and background) 
histograms with full experimental and theoretical systematics, which are the inputs to the likelihood fit. 
The author has derived the systematics uncertainties for the following major backgrounds:
\ttbar, \ZHF\ and single-top. The author has also contributed to the estimation of the fake-\tauhad\
background and its uncertainties. 
Finally, the author implemented and validated a reweighting method in order to scan
the non-resonant $HH$ production with in various \kl\ values, and derived the corresponding
uncertainties. 