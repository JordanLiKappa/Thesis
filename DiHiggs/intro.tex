This chapter describes the search for Higgs boson pair production in the
\bbtt\ channel. 
In this channel, one Higgs boson decays to a \bquark\ pair 
and the other to a $\tau$-lepton pair. 
As the two $\tau$-leptons decay either leptonically or hadronically, 
the analysis is divided further into two sub-channels depending on
their decay mode, 
the $b\bar{b}\tau_{\text{lep}}^{\pm}\tau_{\text{had}}^{\mp}$ channel 
(for simplicity, referred to as \bbttlh\ channel or the lephad channel)
where one of the $tau$-leptons decays leptonically and the other decays
hadronically, 
and the \bbtthh\ channel where both of the $tau$-leptons decay hadronically. 
The decay mode of both $tau$-leptons decay
leptonically is not considered in this analysis 
due to its insignificant contribution. 
In this thesis, the author will present his work 
in the \bbttlh\ channel, and will also show the combination results with 
the \bbtthh\ channel.


% one of the $\tau$-leptons decay leptonically and the other decay hadronically. 
The results of this search are interpreted in terms of resonant and 
non-resonant production of the di-Higgs.
For the non-resonant production, upper limits are set on the SM di-Higgs production
cross section and the Higgs self-coupling $\lambda_{HHH}$.
The non-resonant search is also interpreted in terms of the coupling modifiers using a
Higgs Effective Field Theory approach (TODO: link to theory).
For the resonant search, upper limits are set on the resonance production cross section 
as a function of the resonance mass, 
constraining a model with an extended Higgs sector based on two Higgs doublet (2HDM) model
(TODO: link to theory chapter).
The \bbttlh\ and \bbtthh\
combination results are published in Reference~\cite{ATLAS-CONF-2021-030} 
TODO: if paper is published cite the paper
The $\lambda_{HHH}$ result is combined with other final states of the di-Higgs production
including \bbbb\ and \bbyy. 
The result is published in Reference~\cite{ATLAS-CONF-2021-052} TODO: if paper is published cite the combination paper. 

TODO: add a oveview of the strategy 
In the following sections all the aspects of the analysis are discussed. 
The object and event selections are discussed in Section~\ref{sec:DiHiggs:selection}.
The multivariate analysis is described in Section~\ref{sec:DiHiggs:MVA}. 
The background estimation is explained in Section~\ref{sec:DiHiggs:backgroundEstimation}. 
The systematic uncertainties are described in Section~\ref{sec:DiHiggs:systematics}. 
The statistical interpretation is described in Section~\ref{sec:DiHiggs:analysis}
and the results are presented and discussed in Section~\ref{sec:DiHiggs:results}.