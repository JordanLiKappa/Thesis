This chapter describes the search for Higgs boson pair production in the \bbttlh\ channel, 
where one Higgs boson decays to a \bquark\ pair and the other to a $\tau$-lepton pair, where
one $\tau$-lepton decays leptonically and the other $\tau$-lepton decay hadronically.
% one of the $\tau$-leptons decay leptonically and the other decay hadronically. 
This analysis is the main focus of this thesis work and it is part of the ATLAS Run 2
di-Higgs to \bbtt\ search, published in Reference TODO: add in the publishing dihiggs paper. 
The ATLAS \bbtt\ analysis is performed in two channels depending on the $\tau$-leptons decay modes,
the \bbttlh channel which is described in this thesis, 
and the \bbtthh\ channel where both $\tau$-leptons decay hadronically. 
% The \bbttlh\ channel is presented here as the complete analysis in this channel was developed in this thesis work,
The combination of the two sub-channels is also performed 
in this thesis work and is presented in this chapter.
The results of this search are interpreted in terms of an upper limit on the SM di-Higgs production
cross section and limits on the triple Higgs self-coupling $k_\lambda = \lambda_{HHH}/\lambda_{SM}$ 
for the non-resonant search. 
The non-resonant search is also interpreted in terms of the coupling modifiers using an 
effective field theory approach.
Moreover, the data are also analysed to search for resonances decaying to 
a pair of Higgs bosons and interpreted in terms of upper limits on the resonance
production cross section as a function of the resonance mass, 
constraining a model with an extended Higgs sector based on two doublets (2HDM) model.
The non-resonant and resonant search in the \bbttlh\ channel is part of the ATLAS \bbtt\ 
analysis, as combined with the \bbtthh\ where both $\tau$-leptons decay hadronically.
The combine results are published in References TODO: cite the paper.
The $k_\lambda$ result is combined with other final states of the di-Higgs production
including \bbbb\ and \bbyy\ where the di-Higgs decay to two pairs of b quarks and 
a pair of b quarks with a pair of photons. 
The result is published in Reference TODO: cite the combination paper. 

In the following sections all the aspects of the analysis are discussed. The simulated signal and background
samples are listed in Section 6.1. The object and event selections are discussed in Section 6.2 and 6.3
respectively. The background estimation is explained in Section 6.4. The multivariate analysis is
described in Section 6.5. The list of systematic uncertainties is given in Section 6.6. The statistical
interpretation is described in Section 6.7 and the results are presented and discussed in Section 6.8.