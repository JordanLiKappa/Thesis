It is necessary to correctly identify which $b$'s and $\tau$'s originate from the same LQ. Several approaches are explored in order to pair the $b$-jets 
and $\tau$ leptons correctly. The investigated mass pairing strategies are the following: 

\begin{itemize}
    \item $\mathrm{min}|\Delta m|$: choose the $b\tau$ pairs which minimise the mass difference between the two pairs
    \item $\mathrm{max}|\Delta \phi|$: choose the $b\tau$ pairs which maximise the sum of $\Delta\phi(b,\tau)$ of the two pairs
    \item $\mathrm{min}|\pi-\Delta R|$: choose the $b\tau$ pairs which maximise the sum of $\Delta R(b,\tau)$ of the two pairs.
\end{itemize}

The efficiencies for the different strategies are shown in the~\Cref{fig:LQ_pairing} for the \lephad channel.
This result is derived using the MC truth information as discussed in Ref.~\cite{bib:lq_pairing}, a similar result is expected for the \hadhad channel.
All methods give approximately the same cross-section upper limits as shown in~\Cref{appendix:LQmasspairing},
since energy scale information as $s_T$, LQ mass used as PNN inputs are much more discriminating power than 
other information as the correctness of b$\tau$ pairing. The $\mathrm{min}|\Delta m|$ pairing method is selected as default strategy for 
this analysis as it was the case in the previous paper analysis.

\begin{figure}
  \centering
  \includegraphics[width=0.5\textwidth]{figures/selection/lq3/LQ_pairing.png}
  \caption{Efficiency of pairing the final state b-quarks and $\tau$ leptons for the \lephad channel. Taken from Ref.~\cite{bib:lq_pairing}.}
  \label{fig:LQ_pairing}
\end{figure}
