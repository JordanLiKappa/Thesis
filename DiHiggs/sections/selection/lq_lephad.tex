Event selection is applied to select events compatible with containing a $\ell b\tau_{\mathrm{had}}b+E_{T}^{\mathrm{miss}}$ final state. 
This selection forms the set of events that are used as input to the MVA analysis.
Although the basic concept of the event selection is almost same as for the HH analysis, the LQ analysis focuses on higher energy region.
Therefore, some energy thresholds have been increased with respect to the HH analysis.

First, LQ \lephad channel uses events triggered by SLT (Single Lepton Triggers).
Events for higher energy region are selected by applying selection on \tauhad \pT, \MET and $s_T$.
$s_T$ is the scalar sum of \MET, two highest jets \pT, \tauhad \pT and lepton \pT.
These selections are also helpful to reject multijet background events.
Events must meet the requirements shown in Table~\ref{tab:lq_lephad_event_selection}.

\begin{table}[!ht]
  \centering
  \begin{tabular}{l|c} 
    \hline\hline
                                             & Criteria \\ \hline
    Trigger                                  & Single Lepton Trigger \\
    Light lepton $p_T$                       & $>$ trigger threshold + 1 GeV \\
    $N_{\ell}$                               & $=1$ \\
    $N_{\mathrm{jets}}$                      & $\geq 2$ \\
    $N_{\mathrm{b-jets}}$                    & $=1$ and $=2$ (DL1r 77\% WP)\\
    $\tauhad~p_T$                            & $>100$ GeV \\
    $\tauhad~|\eta|$                         & $<2.3$ \\
    Leading jet $p_T$                        & $>60$ GeV \\
    Sub-leading jet $p_T$                    & $>20$ GeV \\
    $\ell$ and $\tau_{\mathrm{had}}$ charges & Opposite sign\\
    $s_T$                                    & $> 600$ GeV \\
    $E_{T}^{\mathrm{miss}}$                  & $> 100$ GeV \\
    \hline\hline
  \end{tabular}
  \caption{Event selection criteria for the LQ \lephad channel.}
  \label{tab:lq_lephad_event_selection}
\end{table}

%% Thresholds for transverse momentum of light leptons, transverse momentum of jets, number of jets are the same as for the previous early Run-2 analysis.
%% For the analysis with the full Run-2 dataset, the following requirements are tuned:
%% \begin{description}
%%   \item[$\tau_{\mathrm{had}} p_T$]\mbox{}
%%     Tighter \pT requirement is used to target a higher energy region.
%%   \item[$s_T$]\mbox{}
%%     In the previous paper, the $s_T$ requirement did not used because the target energy region was lower.
%%     To aim the higher energy region, the threshold is added. Furthermore, this selection is also useful to reject a lot 
%%     of backgrounds as discussed in Sec~\ref{sec:bkgLQ}.
%%   \item[$E_T^{\mathrm{miss}}$]\mbox{}
%%     In the previous paper, this $E_T^{\mathrm{miss}}$ cut did not used because the target energy region was lower.
%%     To aim the higher energy region, the threshold was added. Furthermore, this selection is also useful to reject a lot of 
%%     backgrounds as discussed in Sec~\ref{sec:bkgLQ}.
%% \end{description}


The event yields after applying the requirements are shown in Table~\ref{tab:lq_lephad_prefit_event_yields} for backgrounds, signal and data.
Plots of basic kinematic distributions are shown in~\Cref{fig:lq_lephad_kinvars1},~\ref{fig:lq_lephad_kinvars2}.
% Although individual kinematic variables do not have enough power to discriminate the signal and background events, a conservative blinding strategy is taken.
% In bins with $S/\sqrt{N}>0.1$ no data is shown and the grey coloured area is the blinded region in the ratio plot.
For \pT, \MET and $s_T$, quadratic sum of $S/\sqrt{B}$ is calculated from left to right, and when it exceeds 0.5, data is blinded from that bin.
Here signal sample of $m_{LQ} = 1100 GeV$ is considered.
More details about the cut optimisation studies can be found in Appendix~\ref{subsec:appendix_lq_criteria_optimization}.


\begin{table}[!htb]
  \centering
  \begin{tabular}{c|c}
    \hline
    \hline
    Source        & Yields \\ \hline
    VH            &     0.7 \\
    WJets         &     1.4 \\
    Diboson       &     4.5 \\
    Zll           &     6.4 \\
    Z$\tau\tau$   &   122.7 \\
    single top    &   224.2 \\
    \ttbar        &  2265.9 \\
    Fake(\ttbar)  &   165.0 \\
    Fake(single-top)  &   4.5 \\
    Fake(others)  &    26.6 \\
    \hline
    \hline
  \end{tabular}
  \caption{Pre-fit event yields in the LQ \lephad signal region.}
  \label{tab:lq_lephad_prefit_event_yields}
\end{table}

\begin{figure}
  \centering
  \includegraphics[width=0.4\textwidth]{figures/selection/lq3/lephad/Kinematics/FinalPlots_TauLH_SR_cluster1_btag77_fullrun2_20210217_BasicKinematics_OS_combined_Jet0Pt_Rebin20.pdf} 
  \includegraphics[width=0.4\textwidth]{figures/selection/lq3/lephad/Kinematics/FinalPlots_TauLH_SR_cluster1_btag77_fullrun2_20210217_BasicKinematics_OS_combined_Jet1Pt_Rebin20.pdf} \\
  \includegraphics[width=0.4\textwidth]{figures/selection/lq3/lephad/Kinematics/FinalPlots_TauLH_SR_cluster1_btag77_fullrun2_20210217_BasicKinematics_OS_combined_Jet0Eta_Rebin15.pdf} 
  \includegraphics[width=0.4\textwidth]{figures/selection/lq3/lephad/Kinematics/FinalPlots_TauLH_SR_cluster1_btag77_fullrun2_20210217_BasicKinematics_OS_combined_Jet1Eta_Rebin15.pdf} \\
  \includegraphics[width=0.4\textwidth]{figures/selection/lq3/lephad/Kinematics/FinalPlots_TauLH_SR_cluster1_btag77_fullrun2_20210217_BasicKinematics_OS_combined_Tau0Pt_Rebin10.pdf} 
  \includegraphics[width=0.4\textwidth]{figures/selection/lq3/lephad/Kinematics/FinalPlots_TauLH_SR_cluster1_btag77_fullrun2_20210217_BasicKinematics_OS_combined_Tau0Eta_Rebin15.pdf} \\
  \caption{Kinematic variable distributions in the signal region. The grey colored band show the blinded region. Here, bins where the quadratic sum of $S/\sqrt{B}$ is greater than 0.2 are blinded for the 1100 GeV signal. $S$ is the number of signal yields , and $\sqrt{B}$ is the square of the number of backgrounds in the bin respectively.}
  \label{fig:lq_lephad_kinvars1}
\end{figure}

\FloatBarrier

\begin{figure}
  \centering
  \includegraphics[width=0.4\textwidth]{figures/selection/lq3/lephad/Kinematics/FinalPlots_TauLH_SR_cluster1_btag77_fullrun2_20210217_BasicKinematics_OS_combined_Lepton0Pt_Rebin10.pdf}
  \includegraphics[width=0.4\textwidth]{figures/selection/lq3/lephad/Kinematics/FinalPlots_TauLH_SR_cluster1_btag77_fullrun2_20210217_BasicKinematics_OS_combined_Lepton0Eta_Rebin15.pdf} \\
  \includegraphics[width=0.4\textwidth]{figures/selection/lq3/lephad/Kinematics/FinalPlots_TauLH_SR_cluster1_btag77_fullrun2_20210217_BasicKinematics_OS_combined_sT_Rebin40.pdf} 
  \includegraphics[width=0.4\textwidth]{figures/selection/lq3/lephad/Kinematics/FinalPlots_TauLH_SR_cluster1_btag77_fullrun2_20210217_BasicKinematics_OS_combined_MET_Rebin10.pdf} \\
  \caption{Kinematic variable distributions in the signal region. The grey colored band show the blinded region. Here, bins where the quadratic sum of $S/\sqrt{B}$ is greater than 0.2 are blinded for the 1100 GeV signal. $S$ is the number of signal yields , and $\sqrt{B}$ is the square of the number of backgrounds in the bin respectively.}
  \label{fig:lq_lephad_kinvars2}
\end{figure}

\FloatBarrier
