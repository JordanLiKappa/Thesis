Electrons are selected and identified by imposing requirements on measured track properties, the shape
of the cluster of energy deposits in the calorimeter, as well as on track-to-cluster matching and the
quality of the track \cite{PERF-2013-03}. 

Electron candidates are identified using a likelihood technique and required to
pass a `loose' working point, which, in combination with additional track hit requirements, provides an
electron efficiency of 95\%. Furthermore, candidates are required to have \pT > 7 GeV where the \pT is
calculated from the four-vector of an electron based on its calorimetric energy and the direction of its track
and cluster) and be within | $\eta$ | < 2.47, excluding the transition region 
between the barrel and end-caps in the LAr calorimeter (1.37 < | $\eta$ | < 1.52).

Electrons are required to be isolated by imposing the `loose' isolation
working point requirement. This isolation requires no near-by tracks or calorimetry energy deposits within
a \pT-dependent variable-size $\Delta{R}$ cone around the electron, achieving 99\% efficiency that is constant across
the entire \pT spectrum. This isolation requirement is inverted to provide control regions for estimating
backgrounds.

When applying the electron tau triggers, the electron isolation working point is changed to `tight'
since the single-leg electron trigger SFs are not available for loose iso electrons.