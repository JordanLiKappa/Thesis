After the event is reconstructed, an overlap-removal procedure is applied to
resolve ambiguities when a physical object is reconstructed as multiple
particles in the ATLAS detector. The angular distance $\dRy =
\sqrt{\Delta y^2 + \Delta \phi^2}$ is used to measure the overlap of two
reconstructed objects.

Overlaps between most of the detector objects used in the analysis are resolved by using 
the standard overlap removal tools \verb|AssociationUtils|~\cite{olrAssociationUtils}. This is not
the case only when resolving overlaps between the reconstructed \tauhadvis, anti-\tauhadvis\ objects and jets,
where an analysis-specific procedure is used, as will be explained in the following.
The \verb|AssociationUtils| tool allows for choosing a working point~\cite{olrWorkingPoints} and
the \textit{Standard} recommended working point is used in the analysis\footnote{The 
\textit{Heavy-flavor} working point (allowing for $b$-jet-aware overlap-removal procedure)  
was also tested and it was found, with respect to the \textit{Standard} working point, 
to give (nearly) identical results in the \hadhad\ (\lephad) channel.}.

The step-by-step procedure that is used to resolve ambiguities in the
reconstructed objects is summarised in the following:
\begin{itemize}
\item $e_1$ - $e_2$: Reject $e_1$ if both electrons share the track and ${\pT}_1
  < {\pT}_2$
\item \tauhadvis - $e$: Reject \tauhadvis if $\dRy < 0.2$ and $e$ passes \verb|DFCommonElectronsLHLoose|
\item \tauhadvis - $\mu$: Reject \tauhadvis if $\dRy < 0.2$:\\
  Case 1 ($\tauhadvis\ \pT > \SI{50}{\GeV}$): $p_{\text{T}, \mu} > \SI{2}{\GeV}$
  and combined muon\\
  Case 2 ($\tauhadvis\ \pT \leq \SI{50}{\GeV}$): $p_{\text{T}, \mu} > \SI{2}{\GeV}$
  \item $\mu$ - $e$: Reject $\mu$ if calo-muon and shared ID track
\item $e$ - $\mu$: Reject $e$ if shared ID track
\item jet - $e$: Reject jet if $\dRy < 0.2$
\item $e$ - jet: Reject $e$ if $\dRy < 0.4$
\item jet - $\mu$: Reject jet if $N_\text{track} < 3$ ($p_\text{T,
  track} > \SI{500}{\MeV}$), and $\dRy < 0.2$
\item $\mu$ - jet: Reject $\mu$ if $\dRy < 0.4$
\end{itemize}
Due to a bug in the PFlow jet reconstruction, an additional OLR step removing
prompt muons being reconstructed as fake jets is implemented in the official ASG
OLR tool.

Additionally, an analysis-specific overlap-removal procedure for \tauhadvis,
anti-\tauhadvis and jets is implemented:
\begin{itemize}
\item jet - \tauhadvis: Reject jet if $\dRy < 0.2$
\item anti-\tauhadvis - jet: Reject anti-\tauhad if jet is \btagged and $\dRy <
  0.2$
\item jet - anti-\tauhadvis: Reject jet  if $\dRy < 0.2$
\end{itemize}
This establishes the following priority: \tauhadvis > \btagged jet >
anti-\tauhadvis > un-tagged jet.

Another priority, \btagged jet > \tauhadvis > anti-\tauhadvis > un-tagged jet, was
investigated as an alternative but found to reduce signal acceptance in the
2-tag region significantly due to limited \tauhad rejection of the DL1r \btag
algorithm at the \SI{77}{\percent} working point\footnote{Additionally, the adopted priority
ensures consistent treatment of electrons, muons and \tauhadvis\ candidates
with respect to $b$-jets in the signal regions of the analysis, which 
further allows for using an $\ell\ell bb$ control region, as will be explained.}. 
With the alternative priority
the signal acceptance is reduced by about \SI{8}{\percent} (\SI{13}{\percent})
in $\taulep\tauhad$ ($\tauhad\tauhad$).