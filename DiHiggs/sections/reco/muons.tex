Muon tracks are reconstructed independently in the inner detector (ID) and the muon spectrometer (MS).
Tracks are required to have a minimum number of hits in each system, and must be compatible in terms
of geometrical and momentum matching. Information from both the ID and MS systems is used in a
combined fit to refine the measurement of the momentum of each muon \cite{PERF-2015-10}. 

Muons are selected with \pT > 7 $\GeV$ and $|\eta| < 2.7$. They are additionally required to pass `loose' identification criteria 
as well as required to pass \code{PflowLoose\_VarRadIso} isolation criteria \cite{isoWP}. The isolation cut is inverted in order to create background control regions.