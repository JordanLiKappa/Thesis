The normalisation of the \ttbar background is included in the final fit as a freely floating nuisance parameter. The \ttbar\ normalisation can be constrained 
strongly by the $Z$+HF CR region.
%The normalisation on the single-top background is from MC cross-section.

Relative acceptance uncertainties are applied on \ttbar in the other regions included in the fit, the \lephad, \hadhad signal region and Z+HF control region. In each signal region also shape variations are checked and applied where they are found to be relevant as described in the following. All these uncertainties are derived by MC-to-MC comparison, as described in Section~\ref{sec:systematics_backgroundmodelling}, following the \href{https://twiki.cern.ch/twiki/bin/view/AtlasProtected/TopMCSystematicsR21}{\underline{recommendations of the Top modelling group}}. 

Uncertainties from the matrix element, parton shower are estimated using the differences between the nominal samples and the corresponding variation samples. 
Uncertainties from ISR are estimated using internal weights which is applied to the alternative samples.
Uncertainties due to PDF, $\alpha_s$, FSR are evaluated using internal alternative weights present in all the \ttbar nominal samples. 
The MC samples to estimate the uncertainties on \ttbar are summarized in Table~\ref{tab:ttbar_alternative_samples}.
%The variations are evaluated by~\Cref{eq:acceptance_unc}. 

\begin{table}[]
  \tiny
  \centering
  \begin{tabular}{ccccccccccccccccc} 
    \hline\hline
    \multirow{2}{*}{Precision} & \multirow{2}{*}{Generators} & \multirow{2}{*}{Parameter} & \multicolumn{2}{c}{ME} & \multicolumn{2}{c}{PS} & \multicolumn{2}{c}{ISR(up)} & \multicolumn{2}{c}{ISR(down)} & \multicolumn{2}{c}{FSR(up)} & \multicolumn{2}{c}{FSR(down)} & \multicolumn{2}{c}{PDF+$\alpha_s$} \\ 
     & & & nom & var & nom & var & nom & var & nom & var & nom & var & nom & var & nom & var \\ \hline
    AF2           & \POWHEG   + \PYTHIA 8 & $\mathrm{h_{damp}}$=1.5 &\ding{51}&         &\ding{51}&         &\ding{51}&         &\ding{51}&\ding{51}&\ding{51}&\ding{51}&\ding{51}&\ding{51}&\ding{51}&\ding{51}\\
    FullSim       & \POWHEG   + \PYTHIA 8 & $\mathrm{h_{damp}}$=1.5 &         &         &         &         &         &         &         &         &&&&&&\\
    AF2           & \AMCatNLO + \PYTHIA 8 & $\mathrm{h_{damp}}$=1.5 &         &\ding{51}&         &         &         &         &         &         &&&&&&\\
    AF2           & \AMCatNLO + \HERWIG 7 & $\mathrm{h_{damp}}$=1.5 &         &         &         &\ding{51}&         &         &         &         &&&&&&\\
    AF2           & \AMCatNLO + \PYTHIA 8 & $\mathrm{h_{damp}}$=3.0 &         &         &         &         &         &\ding{51}&         &         &&&&&&\\
    \hline\hline
  \end{tabular}
  \caption{The MC samples to estimate the uncertainties on \ttbar.}
  \label{tab:ttbar_alternative_samples}
\end{table}

%% The brief descriptions are summarized here.

%% \begin{description}
%%   \item[Matrix element]\mbox{}\\
%%   \item[Parton shower]\mbox{}\\
%%   \item[Initial State Radiation up/down]\mbox{}\\
%%   \item[Final State Radiation up/down]\mbox{}\\

%% \item[Parton Density Function]\mbox{}\\
    %% PDF uncertainty is estimated by using uncertainty prescriptions for PDF4LHC15\_30 combining CT14, MMHT14, and NNPDF3.0,
    %% which is a symmetric Hessian PDF set with $N_{eig}=30$ eigenvectors.
    %% First of all, the following formula is used to compute the PDF uncertainty:
    %% \begin{equation}
    %%   \delta^{\mathrm{pdf}}\sigma=\sqrt{\sum_{i=1}^{30}\left( \sigma^k - \sigma^0 \right)^2},
    %% \end{equation}
    %% where $\sigma$ is a total inclusive cross-section.
    %% The $\sigma^0$ is the PDF4LHC15 central value (PDFset=90900) and  $\sigma^k$ are the eigenvectors (PDFset=90901–90930).
    %% The formula calculates the relative uncertainty w.r.t. the PDF4LHC central value.
    %% This $\delta$ is applied to nominal PDF sample (NNPDF3.0).
    %% It is recommended that PDF+$\alpha_s$ uncertainties is determined by computing the PDF uncertainty for the central $\alpha_s$, 
    %% computing predictions for the upper and lower values of $\alpha_s$, 
    %% and finally adding results in quadrature.
    %% Two NNPDF3.0 variations with different $\alpha_s$ available:
    %% \begin{itemize}
    %%   \item $\alpha_s$ = 0.117 : PDFset=266000
    %%   \item $\alpha_s$ = 0.119 : PDFset=265000
    %% \end{itemize}
    %% Uncertainty on $\alpha_s$ should be derived as:
    %% \begin{equation}
    %%   \delta^{\alpha_s}\sigma=\frac{\sigma(\alpha_s=0.1195)-\sigma(\alpha_s=0.1165)}{2}
    %% \end{equation}
    %% and combined with the PDF uncertainty:
    %% \begin{equation}
    %%   \delta^{\mathrm{PDF}+\alpha_s}\sigma=\sqrt{(\delta^{\mathrm{pdf}}\sigma)^2+(\delta^{\alpha_s}\sigma)^2}
    %% \end{equation}
%% \end{description}

Relative acceptance normalisation uncertainties are calculated using Eq.~\ref{eq:relative_acceptance_unc} and are found in the different fit regions 
to be as large as reported in Table~\ref{tab:lq_background_modeling_ttbar:ttbar_sys}.

For evaluating the shape uncertainties the differences of the variation samples and the nominal sample are compared using the PNN score distributions. The nominal and alternative samples are run through the same pre-selection and the same PNN classification. 

%% \paragraph{Uncertainties on \ttbar in the \lephad channel}\mbox{}\\

The PNN score comparisons of the nominal and alternative samples for the LQ \lephad channel are shown in Appendix~\ref{subsubsec:lq_appendix_systs_ttbar_lephad}. The results show that the shape differences are found for the ME, PS and ISR 
uncertainties, while there is no shape difference in the FSR and PDF+$\alpha_s$ uncertainties.
Thus the ME, PS and ISR up/down sources are taken into account as \ttbar modelling shape uncertainties.
%The relative acceptance normalisation uncertainties are shown in Table~\ref{tab:lq_background_modeling_ttbar:ttbar_sys}.

%% \paragraph{Uncertainties on \ttbar in the \hadhad channel}\mbox{}\\

The PNN score comparisons of the nominal and alternative samples for the LQ \hadhad channel are shown in Appendix~\ref{subsubsec:lq_appendix_systs_ttbar_hadhad}. The results show that the shape differences are found for the ME, PS and ISR/FSR uncertainties.
Thus the ME, PS, ISR up/down and FSR up/down sources are taken into account as modelling shape uncertainties.
%The relative acceptance normalisation uncertainties are shown in Table~\ref{tab:lq_background_modeling_ttbar:ttbar_sys}.

\begin{table}
  \centering
  \small
  \begin{tabular}{c|c|c}
    \hline\hline
    Source         & $\sigma_{\mathrm{ACC}}^{\mathrm{nom}}$ in \lephad SR & $\sigma_{\mathrm{ACC}}^{\mathrm{nom}}$ in \hadhad SR \\ \hline
    ME             &                                & -0.219            \\
    PS             &                                & -0.114            \\
    ISR up/down    &                                &  0.0305, -0.165   \\
    FSR up/down    &                                &  0.0308, -0.0517  \\
    PDF+$\alpha_s$ &                                &  0.149            \\ \hline
    Total up       &                                &              \\
    Total down     &                                &             \\
    \hline\hline
  \end{tabular}
  \caption{Relative normalisation acceptance uncertainties on $t\bar{t}$ in the LQLQ \lephad and \hadhad channels with respect to Z+HF CR.}
  \label{tab:lq_background_modeling_ttbar:ttbar_sys}
\end{table}

% Normalization uncertainty (20210424)
\begin{table}
  \centering
  \small
  \begin{tabular}{c|c|c}
    \hline\hline
    Source         & $\sigma_{\mathrm{ACC}}^{\mathrm{nom}}$ in \lephad SR & $\sigma_{\mathrm{ACC}}^{\mathrm{nom}}$ in \hadhad SR \\ \hline
    ME             &                                & -0.219            \\
    PS             &                                & -0.114            \\
    ISR up/down    &                                &  0.0305, -0.165   \\
    FSR up/down    &                                &  0.0308, -0.0517  \\
    PDF+$\alpha_s$ &                                &  0.149            \\ \hline
    Total          &                                &  0.292, -0.336    \\
    \hline\hline
  \end{tabular}
  \caption{Relative normalisation acceptance uncertainties on $t\bar{t}$ in the LQLQ \lephad and \hadhad channels with respect to Z+HF CR.}
  \label{tab:lq_background_modeling_ttbar:ttbar_sys}
\end{table}

%% Normalization Uncertainty
%% \begin{table}
%%   \centering
%%   \small
%%   \begin{tabular}{c|c|c}
%%     \hline\hline
%%     Source         & $\sigma_{\mathrm{ACC}}^{\mathrm{nom}}$ in \lephad SR & $\sigma_{\mathrm{ACC}}^{\mathrm{nom}}$ in \hadhad SR \\ \hline
%%     ME             & -0.219                                &  0.126            \\
%%     PS             & -0.135                                &  0.0693           \\
%%     ISR up         & 0.351                                 &  0.0196           \\
%%     ISR down       & -0.254                                & -0.137            \\
%%     FSR up         & -0.081                                &  0.0339           \\
%%     FSR down       & 0.065                                 & -0.0598           \\
%%     PDF+$\alpha_s$ & 0.116                                 &  0.141            \\ \hline
%%     Total up       & 0.458                                 &  0.419            \\
%%     Total down     & -0.385                                 & -0.444           \\
%%     \hline\hline
%%   \end{tabular}
%%   \caption{Relative normalisation acceptance uncertainties on $t\bar{t}$ in the LQLQ \lephad and \hadhad channels.}
%%   \label{tab:lq_background_modeling_ttbar:ttbar_sys}
%% \end{table}
