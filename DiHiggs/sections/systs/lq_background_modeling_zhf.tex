The normalisation of the $Z$+HF background is estimated from data through a freely-floating parameter in the final fit. This normalisation is determined from a dedicated $Z$+HF control region.

Relative acceptance uncertainties are applied on $Z$+HF in the other regions included in the fit (the $\tau_{had}\tau_{had}$ signal region and the $\tau_{lep}\tau_{had}$ signal region) to take into account potential differences in the normalisation between the different regions. In each signal region, shape variations are checked as described in the following.  All these uncertainties are derived by MC-to-MC comparison, same as described in Section~\ref{sec:systematics_backgroundmodelling}, following the \href{https://twiki.cern.ch/twiki/bin/viewauth/AtlasProtected/PmgWeakBosonProcesses}{\underline{recommendations of the PMG group for Weak Boson processes}}. 

For the background contribution from $Z$+HF, separate samples for $Z$+bb, $Z$+bc, $Z$+cc are included while $Z$+l, $Z$+bl, $Z$+cl are not included. This evaluation is performed with same samples as for the HH analysis.

Relative acceptance normalisation uncertainties are calculated from the single channel acceptances using Equation~\ref{eq:relative_acceptance_unc} and are reported for each region in Table~\ref{sec:systs:tab:systematics_normalisations_ZHF_LQ}.

\begin{table}
\centering
\small
\begin{tabular}{|c|c|c|}
\hline
Source & Size in \lephad SR & Size in \hadhad SR\\
\hline
Scales         &  &  0.321, -0.214 \\
CKKW           &  &  0.0259 \\
QSF            &  & -0.0337 \\
PDF+$\alpha_s$ &  &  0.0357 \\
PDF choice     &  &  0.00909 \\ \hline
Total          &  &  0.326, -0.221\\
\hline
\end{tabular}
\caption{Relative acceptance normalisation uncertainties on $ZHF$ in the LQLQ \lephad and \hadhad channels with respect to Z+HF CR.}
\label{sec:systs:tab:systematics_normalisations_ZHF_LQ}
\end{table}

%% Normalization Uncertainty
%% \begin{table}
%% \centering
%% \small
%% \begin{tabular}{|c|c|c|}
%% \hline
%% Source & Size in LepHad & Size in HadHad SR\\
%% \hline
%% Scales         &  &  +0.214, -0.320 \\
%% CKKW           &  &  0.0262\\
%% QSF            &  & -0.0379\\
%% PDF+$\alpha_s$ &  &  0.0320 \\
%% PDF choice     &  & -0.00340 \\
%% Total          &  & +0.221, -0.325\\
%% \hline
%% \end{tabular}
%% \caption{Relative acceptance normalisation uncertainties on $Z$+jets.}
%% \label{sec:systs:tab:systematics_normalisations_ZHF_LQ}
%% \end{table}

The PNN score comparisons of the nominal and alternative samples for the LQ \lephad and LQ \hadhad channel are shown in~\Cref{subsubsec:lq_appendix_systs_zhf_lephad} and~\ref{subsubsec:lq_appendix_systs_zhf_hadhad}.
The results show that the shape differences are small for all the uncertainties.
Only relative acceptance normalisation uncertainty is taken into account as Z+HF modelling uncertainties.

% \textcolor{red}{Fill parametrization in lephad with plots.}

%% Shapes of PNN score distribution is evaluated for all the mass points of PNN scores and all PDF and scale variations of the $Z$+jets samples and found to be negligible. This is shown in Fig.~\ref{fig:hadhad_ZHF_CT_MURMUF}.

%% \begin{figure}[!ht]
%% \centering
%% \subfloat[]{\includegraphics[width=.41\textwidth]{figures/systs/LQ/CT14_incl_PNN_1400.pdf}}\quad
%% \subfloat[]{\includegraphics[width=.41\textwidth]{figures/systs/LQ/MUR05MUF05_incl_PNN_1400.pdf}}\quad
%% \caption{Comparison of nominal with different PDF set (left) and scale (right).}
%% \label{fig:hadhad_ZHF_CT_MURMUF}
%% \end{figure}

