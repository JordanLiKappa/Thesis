Acceptance uncertainties on the LQ signals are evaluated by applying alternative weights for these settings for PDF and scale variations
%They are checked applying alternative weights for these settings.
The on-the-fly MadGraphControl framework is used to generate LHE files including alternative weights. For PDF and $\alpha_s$ variations, the NNPDF30\_ nlo\_s\_ 0118 uncertainty sets, consisting of 100 eigenvectors parameterising the uncertainties for all the PDFs and 1 parameter for the $\alpha_s$ variations, are used and are compared to the nominal NNPDF30\_ nlo\_s\_ 0118 to evaluate the size of the variations. % (as recommended by the Physics Modelling Group even if the nominal PDF used in the nominal sample is the NNPDF23\_ lo\_ as\_ 0130\_ qed set).
These variations are combined following the recommendations~\cite{Butterworth:2015oua}, consisting in evaluating the PDF uncertainty $\delta_{PDF}$ by calculating the square root of the sum in quadrature of the 100 variations, evaluating the $\alpha_s$ uncertainty by taking the half difference between the same PDF set with two different $\alpha_s$ values, and then adding in quadrature the two contributions of $\delta_{PDF}$ and $\alpha_s$ uncertainties to obtain the total uncertainty:
\begin{equation}
    \delta^{\mathrm{pdf}}\sigma=\frac{\sqrt{\sum_{i=1}^{100}\left( \sigma^k - \sigma^0 \right)^2}}{nPDF},
\label{eq:PDFunc_LQSignal}
\end{equation}
where $\sigma$ is a total inclusive cross-section.
The $\sigma^0$ is the NNPDF3.0 central value (PDFset=260000) and  $\sigma^k$ are the eigenvectors (PDFset=260001, 260100).
This $\delta$ is applied to nominal PDF sample (NNPDF3.0).
It is recommended that PDF+$\alpha_s$ uncertainties is determined by computing the PDF uncertainty for the central $\alpha_s$, 
computing predictions for the upper and lower values of $\alpha_s$, 
and finally adding results in quadrature.
%% Two NNPDF3.0 variations with different $\alpha_s$ available:
%% \begin{itemize}
%%   \item $\alpha_s$ = 0.117 : PDFset=266000
%%   \item $\alpha_s$ = 0.119 : PDFset=265000
%% \end{itemize}
Uncertainty on $\alpha_s$ is derived as:
\begin{equation}
  \delta^{\alpha_s}\sigma=\frac{\sigma(\alpha_s=0.119)-\sigma(\alpha_s=0.117)}{2}
\end{equation}
and combined with the PDF uncertainty:
\begin{equation}
  \delta^{\mathrm{PDF}+\alpha_s}\sigma=\sqrt{(\delta^{\mathrm{pdf}}\sigma)^2+(\delta^{\alpha_s}\sigma)^2}
\end{equation}

Uncertainties on the LQ signal modelling related to the choice of renormalisation and factorisation scales are evaluated using event weights included in the 
MadGraph samples, varying the scales either together or independently up and down by a factor of two, leading to 7-point scale variations, ($\mu_R$, $\mu_F$) = 
(0.5, 0.5), (1, 0.5), (0.5, 1), (1, 1), (2, 1), (1, 2), (2, 2). The scale uncertainty is then given by the maximum shift of the envelope with respect to the nominal, from 
which the envelope around the nominal can be worked out.
The uncertainty from ISR are estimated using truth samples generated privately where radiation is controlled by jobOptions file, up variation for MC15JobOptions\/Pythia8\_A14\_NNPDF23LO\_Var3cUp\_EvtGen\_Common.py, and down variation for MC15JobOptions\/Pythia8\_A14\_NNPDF23LO\_Var3cDown\_EvtGen\_Common.py.
The ISR uncertainty is symmetrised the up/down uncertainties conservatively taking the maximum magnitude of the two.
The uncertainty due to the choice of PDF set is evaluated using the event weights included in the samples. The PDF variations include 100 replicas of the nominal NNPDF3.0 PDF set as well as central values for two different PDF set, MMHT2014nnlo68cl and CT14nnlo, and two NNPDF3.0nnlo $\alpha_s$ variations.  The NNPDF intra-PDF uncertainty is estimated as the standard deviation of the set of 101 NNPDF3.0 sets. The envelope of the differences between the nominal NNPDF set and the other two PDF sets is taken as an additional uncertainty. 

All these uncertainties are derived by MC-to-MC comparison, as described in Section~\ref{sec:systematics_backgroundmodelling}, following the \href{https://twiki.cern.ch/twiki/bin/viewauth/AtlasProtected/LeptoquarksCluster#Theory_uncertainties}{\underline{recommendations of the Leptoquarks Cluster group for Pair Production Leptoquarks}}. 

Normalization acceptance uncertainties are calculated from the single SR channel acceptances for each mass point using~\Cref{eq:acceptance_unc} and reported in~\Cref{sec:systs:tab:systematics_normalisations_LQ_lephad} and~\ref{sec:systs:tab:systematics_normalisations_LQ_hadhad}.

\begin{table}
\centering
\small
\begin{tabular}{|c|c|c|c|c|c|}
\hline
       & PDF choice & Scales & ISR & PDF+$\alpha_s$ & Total \\
\hline
300 GeV  & 0.0510 & 0.153, -0.141 &  0.154, -0.00962 & 0.0200 & 0.224, -0.216 \\
400 GeV  & 0.0520 & 0.121, -0.128 &  0.0451,  0.0413 & 0.0260 & 0.142, -0.148 \\
500 GeV  & 0.0490 & 0.113, -0.126 & -0.0330, -0.0118 & 0.0300 & 0.131, -0.142 \\
600 GeV  & 0.0490 & 0.106, -0.125 & -0.0173, -0.0154 & 0.0380 & 0.124, -0.141 \\
700 GeV  & 0.0500 & 0.111, -0.129 &  0.0719,   0.121 & 0.0450 & 0.177, -0.189 \\
800 GeV  & 0.0470 & 0.116, -0.132 &  0.0543,  0.0224 & 0.0510 & 0.146, -0.159 \\
850 GeV  & 0.0500 & 0.115, -0.132 & -0.0304,  0.0719 & 0.0570 & 0.155, -0.168 \\
900 GeV  & 0.0530 & 0.118, -0.134 & 0.00559,  0.0126 & 0.0600 & 0.143, -0.157 \\
950 GeV  & 0.0470 & 0.112, -0.132 & 0.00573,  0.0544 & 0.0680 & 0.149, -0.165 \\
1000 GeV & 0.0520 & 0.117, -0.135 &  0.0105, -0.0420 & 0.0710 & 0.152, -0.167 \\
1050 GeV & 0.0460 & 0.115, -0.135 &  0.0413,  0.0358 & 0.0840 & 0.155, -0.171 \\
1100 GeV & 0.0590 & 0.121, -0.138 &  0.0980, -0.0800 & 0.0740 & 0.182, -0.194 \\
1150 GeV & 0.0590 & 0.125, -0.140 & -0.00493, 0.0431 & 0.0820 & 0.166, -0.178 \\
1200 GeV & 0.0670 & 0.125, -0.141 & -0.00612, 0.0379 & 0.0860 & 0.170, -0.182 \\
1250 GeV & 0.0650 & 0.132, -0.144 &   0.123,  0.0429 & 0.0910 & 0.212, -0.220 \\
1300 GeV & 0.0590 & 0.130, -0.144 & 0.00490,  0.0416 & 0.0980 & 0.178, -0.189 \\
1350 GeV & 0.0460 & 0.116, -0.138 &  0.0740,  0.0464 &  0.123 & 0.190, -0.204 \\
1400 GeV & 0.0590 & 0.106, -0.133 &  0.0118,  0.0602 &  0.120 & 0.181, -0.198 \\
1450 GeV & 0.0670 & 0.106, -0.135 & 0.0166, -0.00356 &  0.125 & 0.178, -0.197 \\
1500 GeV & 0.0890 & 0.113, -0.138 & 0.00942, 0.00707 &  0.111 & 0.182, -0.198 \\
1550 GeV & 0.0950 & 0.141, -0.152 & 0.00681,  0.0352 &  0.130 & 0.217, -0.224 \\
1600 GeV & 0.0910 & 0.130, -0.146 & -0.0418, -0.0209 &  0.136 & 0.213, -0.223 \\
1700 GeV & 0.0960 & 0.136, -0.150 & -0.0232, -0.0597 &  0.155 & 0.235, -0.244 \\
1800 GeV &  0.108 & 0.118, -0.143 &  0.0162, 0.00540 &  0.189 & 0.248, -0.261 \\
1900 GeV &  0.143 & 0.161, -0.163 & -0.0232,  0.0264 &  0.187 & 0.286, -0.288 \\
2000 GeV &  0.124 & 0.153, -0.158 &       0,  0.0191 &  0.216 & 0.293, -0.296 \\
\hline
\end{tabular}
\caption{Normalization acceptance uncertainties on LQ signals for each mass for \lephad.}
\label{sec:systs:tab:systematics_normalisations_LQ_lephad}
\end{table}

\begin{table}
\centering
\small
\begin{tabular}{|c|c|c|c|c|c|}
\hline
       & PDF choice & Scales & ISR & PDF+$\alpha_s$ & Total \\
\hline
300 GeV  & 0.0473 & 0.115, -0.123  &  -0.0471,   0.0194 & 0.0192 & 0.134, -0.141 \\
400 GeV  & 0.0491 & 0.106, -0.120  &  -0.0720,  -0.0461 & 0.0230 & 0.139, -0.150 \\
500 GeV  & 0.0490 & 0.112, -0.126  & -0.00826,  -0.0215 & 0.0284 & 0.127, -0.140 \\
600 GeV  & 0.0494 & 0.0924, -0.118 &  0.00882, -0.00735 & 0.0388 & 0.112, -0.134 \\
700 GeV  & 0.0485 & 0.115, -0.130  &   0.0158,  -0.0119 & 0.0421 & 0.133, -0.146 \\
800 GeV  & 0.0519 & 0.121, -0.134  &   0.0115,  -0.0346 & 0.0502 & 0.145, -0.156 \\
850 GeV  & 0.0467 & 0.120, -0.134  &   0.0368,   0.0461 & 0.0598 & 0.149, -0.161 \\
900 GeV  & 0.0513 & 0.115, -0.133  &   0.0101,        0 & 0.0564 & 0.138, -0.154 \\
950 GeV  & 0.0519 & 0.121, -0.136  &  -0.0511,  -0.0238 & 0.0629 & 0.155, -0.167 \\
1000 GeV & 0.0456 & 0.109, -0.130  &   0.0352,   0.0510 & 0.0663 & 0.145, -0.161 \\
1050 GeV & 0.0530 & 0.116, -0.134  &  -0.0538,  -0.0303 & 0.0683 & 0.154, -0.168 \\
1100 GeV & 0.0536 & 0.116, -0.135  &  -0.0603,   -0.107 & 0.0759 & 0.183, -0.196 \\
1150 GeV & 0.0618 & 0.118, -0.136  &   0.0548,   0.0467 & 0.0799 & 0.165, -0.178 \\
1200 GeV & 0.0594 & 0.123, -0.139  & -0.00116,   0.0534 & 0.0893 & 0.172, -0.184 \\
1250 GeV & 0.0622 & 0.131, -0.144  &  -0.0915,  -0.0468 & 0.0890 & 0.193, -0.202 \\
1300 GeV & 0.0660 & 0.124, -0.141  &  -0.0303,  0.00562 & 0.0931 & 0.171, -0.184 \\
1350 GeV & 0.0683 & 0.114, -0.137  & -0.00224,  -0.0213 &  0.108 & 0.173, -0.189 \\
1400 GeV & 0.0915 & 0.135, -0.147  &  -0.0118,  -0.0311 &  0.101 & 0.194, -0.203 \\
1450 GeV & 0.0651 & 0.147, -0.153  &  -0.0202,  -0.0361 &  0.119 & 0.203, -0.208 \\
1500 GeV & 0.0882 & 0.135, -0.146  &   0.0139,   0.0532 &  0.122 & 0.209, -0.216 \\
1550 GeV & 0.0895 & 0.145, -0.152  & -0.00223,   0.0179 &  0.127 & 0.213, -0.218 \\
1600 GeV & 0.0961 & 0.143, -0.152  &   0.0894,   0.0654 &  0.136 & 0.237, -0.243 \\
1700 GeV & 0.0926 & 0.137, -0.149  &  -0.0300,   0.0266 &  0.166 & 0.236, -0.243 \\
1800 GeV &  0.110 & 0.153, -0.159  &   0.0313,  -0.0108 &  0.169 & 0.255, -0.259 \\
1900 GeV &  0.179 & 0.167, -0.165  &   0.0570,  -0.0179 &  0.173 & 0.305, -0.304 \\
2000 GeV & 0.0865 & 0.145, -0.153  &  -0.0776,  -0.0504 &  0.290 & 0.344, -0.348 \\
\hline
\end{tabular}
\caption{Normalization acceptance uncertainties on LQ signals for each mass for \hadhad.}
\label{sec:systs:tab:systematics_normalisations_LQ_hadhad}
\end{table}

The PNN score comparisons of the nominal and alternative samples for the LQ \lephad and LQ \hadhad channel are shown in~\Cref{subsubsec:lq_appendix_systs_signal_lephad} and~\ref{subsubsec:lq_appendix_systs_signal_hadhad}.
The results show that the shape differences are small for all the uncertainties.
Only acceptance normalisation uncertainty for each source is taken into account as modelling shape uncertainties.


%% In the LQLQ $\rightarrow bb\tau_{had}\tau_{had}$ channel the overall PDF+$\alpha_s$ acceptance uncertainty on the normalisation is found to be XX\% for $m_{LQ}= XX$ GeV and XX\% for  $m_{LQ}=XX$ GeV using the NNPDF30\_ nlo\_s\_ 0118 uncertainty sets. 
%% As alternative weights are available also for the alternative PDF sets MMHT2014nlo68clas118 and the CT14nlo, which are also checked. The largest variation comes from the XXXXnlo PDF set and it is of XX\% for $m_{LQ}= XX$ GeV and XX\% for $m_{LQ}= XX$ GeV, so well within the NNPDF30\_ nlo\_s\_ 0118 variations and thus neglected. 
%Alternative weights are available also for the PDF4LHC\_ NLO\_ 30 uncertainty sets which consist of 30 eigenvectors for all the PDFs and 1 parameter for the $\alpha_s$ variations and these are also checked as a cross-check. The PDF+$\alpha_s$ acceptance uncertainty on the normalisation is found to be 0.0055\% for $m_X= 500$ GeV and 0.051\% for  $m_X=1000$ GeV using the PDF4LHC\_ NLO\_ 30 uncertainty sets (smaller than the uncertainties from the NNPDF30\_ nlo\_s\_ 0118 uncertainty sets).
%% The effect of PDF+$\alpha_s$ uncertainties on the shapes of the PNN input variables is also checked and found to be negligible as shown in Appendix~\ref{subsec:appendix_systs_signalsysts}.
%% Thus, the PDF+$\alpha_s$ uncertainties on signal are neglected in the analysis. 

% Scale uncertainties are evaluated using the 7-point scale variations of the renormalisation ($\mu_R$) and the factorisation scale ($\mu_F$) which are combined by taking an envelope of all of the variations (as recommended by the Physics Modelling group). 

%In the LQLQ $bb\tau_{had}\tau_{had}$ channel the scale acceptance uncertainty on the normalisation is found to be XX\% for both $m_{LQ}= XX$ GeV and  $m_{LQ}=XX$ GeV.

%The effect of the scale uncertainties on the shapes of the PNN input variables is also checked and found to be negligible as shown in Appendix~\ref{subsec:appendix_systs_signalsysts}. Thus, the scale uncertainties on signal are neglected in the analysis.
