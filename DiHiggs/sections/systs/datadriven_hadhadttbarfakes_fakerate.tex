The following sources of systematic uncertainties are considered when deriving
the fake rates for the estimation of \ttbar with fake \tauhad in the
$\tauhad\tauhad$ channel: \ttbar normalisation, \ttbar reweighting, non-closure
in \tauhad \pT in the FR measurement region without \tauhad-identification, fake
rate statistical uncertainty, single top and subtraction of other backgrounds. A
summary of the uncertainty sources and their impact on the overall normalisation
is shown in~\Cref{tab:ttbarfakes_hadhad_systnorms}. The full shape information
of these sources is considered in the fit. In the following, the sources of
uncertainties are discussed individually.

\begin{table}[htb]
  \centering
  \begin{tabular}[htb]{lrrr}
    \toprule
    Uncertainty source & ttbarTF & ttbarFT & ttbarFF \\
    \midrule
    \ttbar normalisation & $\substack{+4.0 \%\\-1.1 \%}$ & $\substack{+8.7 \%\\-2.4 \%}$ & $\substack{+12.3 \% \\-3.3 \%}$ \\[0.2em]
    \ttbar reweighting & $\pm 2 \%$ & $\pm 13.1 \%$ & $\substack{+13.2 \% \\ -12.6 \%\\}$\\[0.2em]
    \tauhad \pT non-closure & $\pm 2.4 \%$ &  $\pm 2.8 \%$ & $\pm 5.8 \%$ \\[0.2em]
    Statistical uncertainty & $\pm 5.6 \%$ & $\pm 8.9 \%$ & $\substack{+12.5 \% \\ -12.0 \%}$\\[0.2em]
    Single-top subtraction & $\pm 0.3 \%$ & $\pm 1.0 \%$ & $\pm 1.2 \%$\\[0.2em]
    Other subtraction & $\pm 0.5 \%$ & $\pm 1.3 \%$ & $\pm 1.6 \%$\\
    \midrule
    Total & $\substack{+7.5 \% \\ -6.5 \%}$ & $\substack{+18.4 \% \\ -16.4 \%}$ & $\substack{+22.8 \% \\ -18.7 \%}$ \\
    \bottomrule
  \end{tabular}
  \caption{Impact of fake rate uncertainties on the normalisation of \ttbar with
    fake \tauhad in the $\tauhad\tauhad$ signal region. The contributions are
    separated in fake \tauhad from \ttbar where the subleading \tauhad (TF), the
    leading tau (FT) and both \tauhad are faked by jets (FF).}
  \label{tab:ttbarfakes_hadhad_systnorms}
\end{table}

The \ttbar normalisation in the phase space considered by the analysis is
extracted during the profile likelihood fit extracting the signal. Therefore,
the nominal prefit \ttbar normalisation (1.0) might not accurately describe the
normalisation of the \ttbar background estimate. This will also affect the fake
rate estimation as a considerable amount of \ttbar with true \tauhad needs to be
subtracted when deriving the data-driven estimate of the contribution of fake
\tauhad in the ID and No-ID region used for the fake rate determination. An
estimate of the \ttbar normalisation for the fake rate measurement is derived in
the fake rate CR and conservative uncertainties are applied regarding the
subtraction of non-\ttbar backgrounds and possible missing multijet (up to 2
\%). The extraction of the \ttbar normalisation including the associated
uncertainties is described in~\Cref{subsec:appendix_bkg_ttbar_reweighting}.

\ttbar reweighting, described in~\Cref{subsec:appendix_bkg_ttbar_reweighting},
is applied to events used for the fake rate measurement to correct the modelling
of \ttbar and avoid the oversubtraction of \ttbar with true \tauhad at large
\tauhad transverse momenta leading to unphysical (negative) fake rates. The
reweighting is derived in a region without \tauhad-identification, as a result
no estimate of possible contamination of QCD \tauhad fakes is available. There
are no indications for missing QCD \tauhad fakes in this region. Nevertheless,
the differential contribution of QCD could be enhanced in some phase space
region used for the reweighting. Moreover, it is not clear that the mismodelling
observed in this region can be attributed to \ttbar modelling in isolation as it
could possibly be a combination of \ttbar modelling and other effects affecting
the modelling in this region (e.g.\ reconstruction effects for \tauhad
candidates). Therefore, a conservative approach is taken and the full shape
impact of the reweighting is used as an uncertainty (i.e.\ the up-variation of
the reweighting will recover the nominal, un-reweighted distributions).

After reweighting, the \tauhad \pT distribtions before \tauhad-identification in
the \ttbar fake rate CR still show some residual non-closure (especially 1-prong
\tauhad with \tauhad \pT in $[\SI{40}{\GeV}, \SI{55}{\GeV}]$). Closure plots
before and after reweighting are shown
in~\Cref{fig:ttbarReweighting_modelling_CR}. The full non-closure is applied as
an uncertainty during the fake rate estimation. The size of the non-closure is
depicted in~\Cref{fig:ttbarReweighting_tauPt_closure}.

The statistical uncertainty of the measured fake rates is coherently varied by
one standard deviation when estimating the contribution of fake \tauhad from
\ttbar in the $\tauhad\tauhad$ signal region.

Other backgrounds, primarily single top and V+jets, need to be subtracted for
the measurement of the data-driven fake rates. The subtraction of these
backgrounds is small compared to the subtraction of true \tauhad \ttbar
therefore only normalisation uncertainties are considered. The subtraction of
single top is varied by the single top cross section uncertainty (6\%). The
normalisations of all other backgrounds (predominantly V+jets) are varied by
25\%.

The treatment of uncertainties for the fake scale factor method is briefly
outlined in~\Cref{sec:ttbarfake_hadhad_sf_method}.