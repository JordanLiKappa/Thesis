The normalisation of the $t\bar{t}$ background is estimated from data through a freely-floating parameter included in the final fit. %This normalisation is determined from the low PNN region of the $\tau_{lep}\tau_{had}$ SLT signal region which is dominated by $t\bar{t}$ events.

Relative acceptance uncertainties are applied on $t\bar{t}$ in the other regions included in the fit: 
the \lephad signal region, the \hadhad signal region, and the Z+HF control region.  Relative acceptance 
normalisation uncertainties are calculated from the single channel acceptances using 
Equation~\ref{eq:relative_acceptance_unc} and are recorded in Table~\ref{sec:systs:tab:systematics_normalisations_ttbar_LQ}.

In each signal region, additional sources of shape uncertainty are 
considered and included where they are non-negligible. All these uncertainties are derived by 
MC-to-MC comparison, as described in Section~\ref{sec:systematics_backgroundmodelling}, following the 
\href{https://twiki.cern.ch/twiki/bin/view/AtlasProtected/TopMCSystematicsR21}{\underline{recommendations of the Top modelling group}}. 

\begin{itemize}
\item Uncertainties due to PDF, $\alpha_s$, FSR and scales are evaluated using internal alternative weights present in all the $t\bar{t}$ nominal samples. 
\item Uncertainties from the parton shower, matrix element and ISR are estimated using the differences between the nominal samples and the corresponding variations samples. 
\end{itemize}

For evaluating the shape uncertainties, the differences of the variation samples and the nominal sample are compared 
in the PNN score distributions. For both the nominal and varied samples, the same event selections and PNN classification are applied. Figure.~\ref{fig:ttbarsyst_lephad_isr_PTBB} 
show the comparison of the variation samples and the nominal sample for each systematic source. 
For ME and PS variations, shape difference is taken into account, while for ISR, FSR and PDF+$\alpha_s$ variations, 
the shape difference is concluded to be negligible and is not included in the fit.

\textcolor{red}{Fill parametrization in lephad with plots.}

\begin{table}
\centering
\small
\begin{tabular}{|c|c|c|c|}
\hline
Source & Size in LepHad SR & Size in HadHad SR & Size in Z+HF CR\\
\hline
ME &  & -0.215 & \\
PS &  &  -0.112 & \\
ISR &  & +0.206, -0.165 &  \\
FSR & & +0.0307, -0.0552 & \\
PDF+$\alpha_s$ &  &  &  \\
Total &  &  & \\
\hline
\end{tabular}
\caption{Relative acceptance normalisation uncertainties on $t\bar{t}$.}
\label{sec:systs:tab:systematics_normalisations_ttbar_LQ}
\end{table}


\begin{figure}[!h]
\centering
\subfloat[]{\includegraphics[width=.41\textwidth]{figures/systs/LQ/ME_truth_ttbarpt.pdf}} \quad
\subfloat[]{\includegraphics[width=.41\textwidth]{figures/systs/LQ/MEparam_PNN_900.pdf}} \quad
\caption{Parametrization of the ME variation with truth ttbar pt. Left plot shows truth ttbar distribution and its var/nom ratio, right plot shows the original variation (dashed line) and variation parametrized by ratio of truth ttbar pt distribution (red and blue line) in the PNN distribution for a 900~\GeV resonance.}
\label{fig:ttbarsyst_lephad_isr_PTBB}
\end{figure}


