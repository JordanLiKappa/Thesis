\FloatBarrier

\subsection{Multijet with fake-\tauhad in the \hadhad channel}

The LQ \hadhad\ channel uses the same data-driven fake-factor (FF) method as the di-Higgs \hadhad\ channel to perform the multijet background 
estimate. Details about the di-Higgs \hadhad\ multijet background estimate can be found in~\Cref{subsec:HadHadmultijet}, here only differences 
relevant for the LQ analysis are discussed. The FFs for the LQ \hadhad\ channel are estimated in the SS CR with the LQ \hadhad selection described 
in~\Cref{subsec:sellq_hadhad} applied but without the \MET requirement. The LQ signal region is 1-$b$-tag and 2-$b$-tag inclusive, thus the FFs are 
derived in the 1-$b$-tag and 2-$b$-tag inclusive region as well and the 1-$b$-tag to 2-$b$-tag transfer factor is not needed. The CR to derive the FFs 
in the LQ \hadhad channel is depicted in~\Cref{fig:hadhad_FF_CR}. 

\begin{figure}
\centering
\includegraphics[width=0.3\textwidth]{./figures/bkg/lq_hadhad_multijet/LQ_FF_CR.png}
\caption{Schematic depiction of the application of the fake-factor method used to estimate the multijet background in the LQ \hadhad channel. LQ \hadhad selection described in~\Cref{subsec:sellq_hadhad} without the \MET cut is applied for both regions.}
\label{fig:hadhad_FF_CR}
\end{figure}

In the same way as for the di-Higgs analysis, the FFs are derived separately for 1- and 3-prong \tauhad, separately for the STT and DTT trigger 
categories, and separately for the year of data-taking. Moreover, the FFs are derived separately for leading and sub-leading \tauhad. For the 
DTT events, it is averaged and further separated into \tauhad \pT bins, barrel and endcap with threshold of $|\eta| = 1.52$, while for the STT 
events it is not averaged and not separated into \tauhad \pT bins due to low statistics. The equation to derive FFs are depicted in~\Cref{eq:HH:hadhad:multijet:FFi},~(\ref{eq:HH:hadhad:multijet:FFavg}).

The FFs for the LQ \hadhad channel are given in~\Cref{fig:lq_hadhad_fake_factors}. The FFs for the LQ and HH analyses (shown in~\Cref{fig:hadhad_fake_factors}) are comparable given the differences in the selection. Plots used to derive the FFs are collected 
in~\Cref{subsec:appendix_bkg_lq3_multijet_fakes} for reference. \Cref{fig:lq_app_bkg_hadhad_1b1pDTT} -~\ref{fig:lq_app_bkg_hadhad_2b3pSubleadSTT}  show the composition of regions used for the FF estimate and in particular the contribution 
of non-fake backgrounds in these regions. In addition, a comparison between the FFs derived in 1-$b$-tag-only region versus the FFs 
derived in 1-$b$-tag + 2-$b$-tag inclusive region are shown in~\Cref{fig:lq_app_bkg_hadhad_fake_factors_12tag}, which shows no significant
differences between the FFs. Nevertheless the observed difference is treated as systematic uncertainty in the fit.
The limits derived using the 1-$b$-tag FFs + transfer-factors (\Cref{fig:lq_app_bkg_hadhad_TF}) are the same as limits derived using the 1+2$b$-tag FFs as shown in~\Cref{fig:lq_app_bkg_hadhad_12tag_limit_comparison}.


\begin{figure}
  \centering
  \includegraphics[width=.32\textwidth]{figures/bkg/lq_hadhad_multijet/FF_20210211/ff_FF_NomAll_16_FFBC_152_20210207_TauFFHadHad1D_DTT_FF_3tag_TauPtEndcap_1P.pdf}
  \includegraphics[width=.32\textwidth]{figures/bkg/lq_hadhad_multijet/FF_20210211/ff_FF_NomAll_17_FFBC_152_20210207_TauFFHadHad1D_DTT_FF_3tag_TauPtEndcap_1P.pdf}
  \includegraphics[width=.32\textwidth]{figures/bkg/lq_hadhad_multijet/FF_20210211/ff_FF_NomAll_18_FFBC_152_20210207_TauFFHadHad1D_DTT_FF_3tag_TauPtEndcap_1P.pdf}

  \includegraphics[width=.32\textwidth]{figures/bkg/lq_hadhad_multijet/FF_20210211/ff_FF_NomAll_16_FFBC_152_20210207_TauFFHadHad1D_DTT_FF_3tag_TauPtEndcap_3P.pdf}
  \includegraphics[width=.32\textwidth]{figures/bkg/lq_hadhad_multijet/FF_20210211/ff_FF_NomAll_17_FFBC_152_20210207_TauFFHadHad1D_DTT_FF_3tag_TauPtEndcap_3P.pdf}
  \includegraphics[width=.32\textwidth]{figures/bkg/lq_hadhad_multijet/FF_20210211/ff_FF_NomAll_18_FFBC_152_20210207_TauFFHadHad1D_DTT_FF_3tag_TauPtEndcap_3P.pdf}

  \includegraphics[width=.32\textwidth]{figures/bkg/lq_hadhad_multijet/FF_20210211/ff_FF_NomAll_16_FFBC_152_20210207_TauFFHadHad1D_STT.pdf}
  \includegraphics[width=.32\textwidth]{figures/bkg/lq_hadhad_multijet/FF_20210211/ff_FF_NomAll_17_FFBC_152_20210207_TauFFHadHad1D_STT.pdf}
  \includegraphics[width=.32\textwidth]{figures/bkg/lq_hadhad_multijet/FF_20210211/ff_FF_NomAll_18_FFBC_152_20210207_TauFFHadHad1D_STT.pdf}

  \caption{Fake-factors for 1-prong DTT (top), 3-prong DTT (middle), and STT (bottom) for the data-taking periods 15-16 (left),
           17 (center), and 18 (right) for the LQ \hadhad channel. The black points for DTT are fake-factors derived from barrel region ($|\eta| < 1.52$),
           and the red points from endcap region ($|\eta| > 1.52$).}

  \label{fig:lq_hadhad_fake_factors}
\end{figure}

%Multijet fake-\tauhad contributions in LQ \hadhad SR are very small due to $s_T$ and \MET requirements. The $s_T$ requirement rejects a large fraction of background events in general, and \MET cut further reduces multijet events. First, multijet events are estimated by FF method described in Sec.~\ref{subsec:HadHadmultijet}. In this channel, the FFs are derived from the region where the LQ \hadhad selection without \MET cut is applied, and 0-tag region is used instead of 1 or 2-tag regions due to the lack of statistics and ttbar mis-modeling (described in Appendix.~\ref{subsec:appendix_bkg_ttbar_reweighting}). Even using 0-tag FF, multijet events cannot be estimated especially for higher energy region such as high \MET region, thus multijet distribution in low \MET region ($\MET < 100 $ GeV) is fitted by exponential function and then it is extrapolated to high \MET region to estimate multijet event yields. It is found to be negligible compared to other background events.

%% Sec.~\ref{subsubsection:lq_hadhad_ttbar_mismodeling} discusses modelling of \ttbar\ background, Sec.~\ref{subsubsection:lq_hadhad_fake_factor} discusses the estimation of fake-\tauhad multijet background from the FF derived from 0-tag region, and finally Sec.~\ref{subsubsection:lq_hadhad_high_met_requirement} shows the fake-\tauhad multijet is significantly reduced by the \MET requirement.


%% % --------------------------------------------------------------------- % 
%% \subsubsection{\ttbar modelling in the \hadhad signal region} 
%% \label{subsubsection:lq_hadhad_ttbar_mismodeling}
%% % --------------------------------------------------------------------- % 

%% The $s_T$ distribution in the OS-SR without $s_T>600$ GeV and $\MET>100$ GeV cuts is shown in Figure~\ref{fig:bkg_lq_hadhad:sTnoSTMET}.
%% This plot does not contain the ttbar fake-\tauhad and multijet fake-\tauhad, thus the ratio plot shows a comparison of data and true-\tauhad MC predictions. In the low $s_T$ region the data is significantly higher than the MC prediction. That implies the contribution of fake-\tauhad from \ttbar and multijet is large in this region. However, in the high $s_T$ region the ratio of data to true-\tauhad MC decreases, and in some bins the MC prediction is even higher than the data yields. This region is dominated by the \ttbar\ events and the mis-modelling of the \ttbar\ background is significant.

%% \begin{figure}
%% \centering
%% \includegraphics[width=0.4\textwidth]{./figures/bkg/lq_hadhad_multijet/Merged_TauHH_LQ_1618_No2_noSTMET_20201026_Preselection_OS_incl12tag_sT_Rebin10_logy}
%% \caption{The $s_T$ distributions in the OS signal region with 1- and 2-tag inclusively without the $s_T>600$ GeV and \MET $>$ 100 GeV cuts.}
%% \label{fig:bkg_lq_hadhad:sTnoSTMET}
%% \end{figure}

%% % --------------------------------------------------------------------- % 
%% \subsubsection{Multijet fake-\tauhad estimation with the fake factor method}
%% \label{subsubsection:lq_hadhad_fake_factor}
%% % --------------------------------------------------------------------- % 
%% The multijet background with jets faking \tauhad is evaluated using the FF method as discussed in detail in Section~\ref{subsec:HadHadmultijet}. 
%% Here, only a brief description of the FF method is summarised. The FF method is based on the 
%% following equation:
%% \begin{equation}
%%   N (\mathrm{fake|\tau_{\mathrm{had}}\mathchar`-ID~OS}) = N(\mathrm{fake|anti~\tau_{\mathrm{had}}\mathchar`-ID~OS}) \times \frac{N(\mathrm{fake|\tau_{\mathrm{had}}\mathchar`-ID~SS})}{N(\mathrm{fake|anti~\tau_{\mathrm{had}}\mathchar`-ID~SS})},
%% \end{equation}
%% where $N(\mathrm{fake})$ is the number of multijet fake-\tauhad events, and the OS and SS are opposite-sign and same-sign regions, 
%% respectively. It is assumed that the probability of a jet to fake \tauhad is the same in OS and SS. The second term on the right side is the so-called 
%% fake factor.

%% %Each term is calculated as :
%% %\begin{eqnarray}
%% %  N (\mathrm{fake|\tau_{\mathrm{had}}\mathchar`-ID~OS}) =  N(\mathrm{data}|\tau_{\mathrm{had}}\mathchar`-ID~OS) -  N(\mathrm{MC~with~true}\tau_{\mathrm{had}}|\tau_{\mathrm{had}}\mathchar`-ID~OS) 
%% %  N(\mathrm{fake|anti~\tau_{\mathrm{had}}\mathchar`-ID~OS}) \times \frac{N(\mathrm{fake|\tau_{\mathrm{had}}\mathchar`-ID~SS})}{N(\mathrm{fake|anti~\tau_{\mathrm{had}}\mathchar`-ID~SS})},
%% %\end{eqnarray}

%% % The FFs for the LQLQ \tauhad are calculated in the 1- and 2-btag inclusive SS region, but there is a problem due to \ttbar mis-modeling.
%% % It is hard to calculate the numerator and denominator, because the number of MC yields with true \tauhad is larger than the data yields.
%% % All of those plots can be found in Apendix~\ref{subsec:appendix_bkg_lq3_multijet_fakes}.

%% As discussed in Section~\ref{subsubsection:lq_hadhad_ttbar_mismodeling}, the high $s_T$ region is affected by the mis-modelling of the \ttbar background. Figure~\ref{fig:bk:lq_hadhad_multijet:taupt_os} shows the 
%% leading/sub-leading $\tau$-\pT distribution in the OS 1+2 tag region with $s_T>600$ GeV cut but without the $\MET > 100$ GeV requirement. In these plots, \ttbar with fake-\tauhad backgrounds are derived from the MC fakes described in Section~\ref{subsec:LQLepHadttbarfake}. More plots can be found in~\ref{aux:tauPt_ID_antiID_1and3prong_OSandSS}. Due to the \ttbar mis-modelling and lack of statistics, the number of data events are smaller than the MC background prediction in some bins. %The inclusion of the MC-based fake-\tauhad background leads to a larger data/prediction disagreement. 
%% Thus the FFs cannot be calculated in the 1+2 tag SS region. % and a new region to evaluate the fake estimate needs to be found.

%% \begin{figure}
%%   \centering
%%   \subfloat[DTT 1-prong leading $\tau$-\pT (\tauhad ID)]{\includegraphics[width=0.4\textwidth]{./figures/appendix/lq3/background/multijet_fakes/ff_dist_raw_Merged_TauHH_LQ_1618_No3_noMET_20201026_combined_TauFFHadHad1D_DTT_LeadTauPt1P_tau_OS}}
%%   \subfloat[DTT 1-prong leading $\tau$-\pT (anti-\tauhad ID)]{\includegraphics[width=0.4\textwidth]{./figures/appendix/lq3/background/multijet_fakes/ff_dist_raw_Merged_TauHH_LQ_1618_No3_noMET_antiTau_20201026_combined_TauFFHadHad1D_DTT_LeadTauPt1P_antiTau_OS}}\\
%%   \subfloat[DTT 3-prong leading $\tau$-\pT (\tauhad ID)]{\includegraphics[width=0.4\textwidth]{./figures/appendix/lq3/background/multijet_fakes/ff_dist_raw_Merged_TauHH_LQ_1618_No3_noMET_20201026_combined_TauFFHadHad1D_DTT_SubleadTauPt3P_tau_OS}}
%%   \subfloat[DTT 3-prong leading $\tau$-\pT (anti-\tauhad ID)]{\includegraphics[width=0.4\textwidth]{./figures/appendix/lq3/background/multijet_fakes/ff_dist_raw_Merged_TauHH_LQ_1618_No3_noMET_antiTau_20201026_combined_TauFFHadHad1D_DTT_SubleadTauPt3P_antiTau_OS}}\\
%%   \caption{Leading and sub-leading 1- or 3-prong \tauhad transverse momentum distributions in the LQ \hadhad OS 1+2 $b$-tag inclusive region.
%%     Plots on the left show the distributions in the \tauhad-ID region, and on the right in the anti-\tauhad-ID region.}
%%   \label{fig:bk:lq_hadhad_multijet:taupt_os}
%% \end{figure}


%% \begin{figure}
%%   \centering
%%   \includegraphics[width=0.4\textwidth]{./figures/bkg/lq_hadhad_multijet/Merged_TauHH_LQ_1618_No3_noMET_20201026_Preselection_SS_0tag_Tau0Pt_logy.pdf}
%%   \includegraphics[width=0.4\textwidth]{./figures/bkg/lq_hadhad_multijet/Merged_TauHH_LQ_1618_No3_noMET_20201026_Preselection_SS_0tag_Tau1Pt_logy.pdf}
%%   \caption{The leading and sub-leading $\tau$-\pT distributions in the SS 0-tag region.}
%%   \label{fig:bkg_lq_hadhad:TauPt_withSTMET_SS_0tag}
%% \end{figure}


%% To avoid \ttbar mis-modeling, 0-tag region is used. Figure~\ref{fig:bkg_lq_hadhad:TauPt_withSTMET_SS_0tag} shows the same distributions in the SS 0-tag region after the $s_T$ requirement. 
%% There are much smaller \ttbar events, and the data/MC ratio is larger than 1 (the red arrows show the bins with the ratio 
%% larger than 0.5). The FFs from the 1+2-tag SS region and 0-tag SS region are calculated using the distributions shown in 
%% Figures~\ref{fig:bk:lq_hadhad_multijet:taupt_os} and Figures~\ref{fig:bkg_lq_hadhad:TauPt_withSTMET_SS_0tag}, respectively.

%% \begin{figure}
%%   \centering
%%   \includegraphics[width=0.4\textwidth]{./figures/bkg/lq_hadhad_multijet/ff_1tag.pdf}
%%   \includegraphics[width=0.4\textwidth]{./figures/bkg/lq_hadhad_multijet/ff_0tag.pdf}
%%   \caption{The fake factor distributions as a function of the \tauhad-\pT in the SS region. The plot on the left shows 1+2-tag SS region and 
%%   the plot on the right show the 0-tag SS region.}
%%   \label{fig:bkg_lq_hadhad:ff_hadhad}
%% \end{figure}

%% An example of the FFs for the DTT events is shown in Figure~\ref{fig:bkg_lq_hadhad:ff_hadhad}. The FFs on the left are calculated 
%% using the 1+2-tag SS region, while the FFs on the right are calculated using the 0-tag SS region. The FFs on the left are negative 
%% in some bins, while the FFs on the right are positive. Therefore, the FF derived from the 0-tag SS region is used to estimate multijet events in the SR. The results are summarised in Figure~\ref{fig:bkg_lq_hadhad:TauPt_withSTMET_SS_12tag}. 

%% \begin{figure}
%%   \centering
%%   \subfloat[DTT 1-prong leading $\tau$]{\includegraphics[width=0.4\textwidth]{./figures/bkg/lq_hadhad_multijet/ff_FF_NomAll_1618_sT600_20201006_TauFFHadHad1D_DTT_FF_0tag_TauPt_1P}}
%%   \subfloat[DTT 3-prong leading $\tau$]{\includegraphics[width=0.4\textwidth]{./figures/bkg/lq_hadhad_multijet/ff_FF_NomAll_1618_sT600_20201006_TauFFHadHad1D_DTT_FF_0tag_TauPt_3P}}\\
%%   \subfloat[DTT 1-prong sub-leading $\tau$]{\includegraphics[width=0.4\textwidth]{./figures/bkg/lq_hadhad_multijet/ff_FF_NomAll_1618_sT600_20201006_TauFFHadHad1D_DTT_FF_0tag_SubleadTauPt_1P}}
%%   \subfloat[DTT 3-prong sub-leading $\tau$]{\includegraphics[width=0.4\textwidth]{./figures/bkg/lq_hadhad_multijet/ff_FF_NomAll_1618_sT600_20201006_TauFFHadHad1D_DTT_FF_0tag_SubleadTauPt_3P}} \\
%%   \subfloat[STT 1-prong leading $\tau$]{\includegraphics[width=0.4\textwidth]{./figures/bkg/lq_hadhad_multijet/ff_FF_NomAll_1618_sT600_20201006_TauFFHadHad1D_STT_FF_0tag_TauPt_1P}}
%%   \subfloat[STT 3-prong leading $\tau$]{\includegraphics[width=0.4\textwidth]{./figures/bkg/lq_hadhad_multijet/ff_FF_NomAll_1618_sT600_20201006_TauFFHadHad1D_STT_FF_0tag_TauPt_3P}}\\
%%   \subfloat[STT 1-prong sub-leading $\tau$]{\includegraphics[width=0.4\textwidth]{./figures/bkg/lq_hadhad_multijet/ff_FF_NomAll_1618_sT600_20201006_TauFFHadHad1D_STT_FF_0tag_SubleadTauPt_1P}}
%%   \subfloat[STT 3-prong sub-leading $\tau$]{\includegraphics[width=0.4\textwidth]{./figures/bkg/lq_hadhad_multijet/ff_FF_NomAll_1618_sT600_20201006_TauFFHadHad1D_STT_FF_0tag_SubleadTauPt_3P}}
%%   \caption{Fake factors calculated using the 0-tag SS events in the LQ \hadhad channel.}
%%   \label{fig:bkg_lq_hadhad:TauPt_withSTMET_SS_12tag}
%% \end{figure}



%% % --------------------------------------------------------------------- % 
%% \subsubsection{High \MET requirement}
%% \label{subsubsection:lq_hadhad_high_met_requirement}
%% % --------------------------------------------------------------------- % 

%% Figure~\ref{fig:bkg_lq_hadhad:met_with_multijet_fake} shows the \MET distribution of the MC simulation and the multijet fakes derived from the FF in 0-tag SS
%% region. No transfer factor is applied from 0-tag to 1+2-tag region but it is shown in Section~\ref{subsec:HadHadmultijet} that the transfer factor from 1-tag 
%% to 2-tag is close to 1 ({\color{red} a detailed study on the transfer factor will follow}).

%% \begin{figure}
%%   \centering
%%   \includegraphics[width=0.5\textwidth]{./figures/bkg/lq_hadhad_multijet/Merged_TauHH_LQ_1618_No3_noMET_20201026_Preselection_OS_incl12tag_MET.pdf}
%%   \caption{The \MET distribution with MC-based \ttbar fake-\tauhad's and data-driven multijet fakes. The multijet fakes are estimated using the 0-tag FFs.}
%%   \label{fig:bkg_lq_hadhad:met_with_multijet_fake}
%% \end{figure}

%% The multijet fake-\tauhad distribution in the OS region is shown in Figure~ \ref{fig:bkg_lq_hadhad:met_fit} on the left. A large fraction of multi-jet events are distributed in the low \MET region below 100~GeV, while in the high \MET region above 100 GeV the multijet fake-\tauhad background 
%% have negative yields. Therefore, to estimate the multijet fakes in the high \MET region, the following procedure is applied: 

%% \begin{figure}
%%   \centering
%%   \includegraphics[width=0.45\textwidth]{./figures/bkg/lq_hadhad_multijet/met_only_multijet.pdf}
%%   \includegraphics[width=0.45\textwidth]{./figures/bkg/lq_hadhad_multijet/met_extrapolation.pdf}
%%   \caption{Left: The multijet fake-\tauhad \MET distribution using the data-driven multijet fakes estimated in the 0-tag region. Right: The extrapolated \MET distribution as discussed in the text.}
%%   \label{fig:bkg_lq_hadhad:met_fit}
%% \end{figure}


%% \begin{itemize}
%%   \item Fit the low \MET ($\MET<100$ GeV) region by the exponential function
%%   \item Extrapolate the function to the high \MET ($\MET> 100$ GeV) region
%%   \item Extract the expected multijet fake event yields
%% \end{itemize}

%% The fit and the extrapolated yields are shown in Figure~\ref{fig:bkg_lq_hadhad:met_fit} on the right. The extrapolated multijet yields are very small compared with other background. Therefore, the multijet background are considered as negligible in the following sections.
