\subsection{Multijet with fake-\tauhad in the \lephad channel}

Multijet events included in the LQ \lephad channel are estimated to be negligible after application of the \pT(\tauhad) > 100 GeV, $s_T$ and \MET requirements. \Cref{fig:lq_lephad_rQCD} shows the rQCD distribution for 1-$b$-tag region estimated using the HH selection, described in~\Cref{subsec:LepHadfake}. The 2-$b$-tag region is shown in~\Cref{fig:SLT_rQCD}. The rQCD for \tauhad \pT > 100 GeV are almost 0, and the additional $s_T$ and \MET requirements further reduces the multijet fakes.

\begin{figure}
  \centering
  \includegraphics[width=.45\textwidth]{figures/bkg/lq_lephad_multijet/rQCD_SLT_1p_1tag_mutau.png}
  \includegraphics[width=.45\textwidth]{figures/bkg/lq_lephad_multijet/rQCD_SLT_1p_1tag_eltau.png}\\
  \includegraphics[width=.45\textwidth]{figures/bkg/lq_lephad_multijet/rQCD_SLT_3p_1tag_mutau.png}
  \includegraphics[width=.45\textwidth]{figures/bkg/lq_lephad_multijet/rQCD_SLT_3p_1tag_eltau.png}\\
\caption{$\mathrm{r}_{\mathrm{QCD}}$ for 1-b-tag region, 1-prong (left) and 3-prong (right) \tauhad candidates for $e\tauhad$ channel (top) and $\mu\tauhad$ (bottom) when requiring same-sign lepton-tau pairs for the \lephad SLT category.}
  \label{fig:lq_lephad_rQCD}
\end{figure}

