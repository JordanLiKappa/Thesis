In the \hadhad\ channel, the multijet background is estimated using a data-driven fake-factor method, following the method used in the previous round of this analysis~\cite{HIGG-2016-16}. Fake-factors ($FF$s) are derived in a control region consisting of events with two \tauhadvis\ that have same-sign (SS) electric charges (SS CR), as ratios of the number of events with two loose \tauhad\ to the number of events with one loose and one anti-\tauhad\ candidate\footnote{Due to the skimming applied in the ``HIGG4D3'' derivations (where at least one loose \tauhad\ is required), the \tauhad-ID can be inverted only for one of the \tauhad.}. These $FF$s are applied to events with one loose and one anti-\tauhad\ candidate that have opposite-sign (OS) electric charges, in order to predict the number of OS multijet events with two loose \tauhad.

Fake-factors are derived separately for 1- and 3-prong \tauhad, separately for
the STT and DTT trigger categories, and separately for the 0-, 1- and 2-$b$-tag
regions. Moreover, the fake-factors are split by year of data-taking to account
for different \tauhad identification in the HLT and topologies being selected by
the changing trigger selection. However, due to low statistics in the SS
2-$b$-tag region, $FF$s derived in the 1-$b$-tag region are applied in the
2-$b$-tag region. For that reason, a set of transfer-factors ($TF$s) is used to
correct the multijet normalisation in the 2-$b$-tag region, as discussed below. 
The 0-$b$-tag regions are used only for validation purposes. 
The different regions used for the multijet estimation are schematically
depicted in Figure~\ref{fig:hadhadABCDMethod}. 


\begin{figure}[!h]
\centering
\captionsetup[subfigure]{justification=centering}
\subfloat[]{
\includegraphics[height=3.4cm]{figures/bkg/hadhad_multijet/FFABCD_note.pdf}
}
\subfloat[]{
\includegraphics[height=3.4cm]{figures/bkg/hadhad_multijet/FFdiagram_note.pdf}
}
        \caption{Schematic depiction of the application of the fake-factor method used to estimate the multijet background in the di-Higgs \hadhad\ channel. (\textbf{a}) Fake factors, $FF$s, are calculated in the SS region as $C/D$, after subtracting the number of simulated non-multijet events from the number of the data events in all regions. These fake-factors are then applied to the OS-region events that contain one anti-\tauhad\ candidate, region $B$, to obtain a multijet estimation in the OS region where 2 loose \tauhad\ are required, region $A$. (\textbf{b}) Fake-factors are measured separately in regions with 0, 1 and 2 $b$-tagged jets; however, fake-factors measured in the 1-$b$-tag region are applied in the 2-$b$-tag region. A set of transfer-factors, $TF$, between the 1- and 2-$b$-tag regions is derived and applied as a normalisation correction for the multijet background estimation in the 2-$b$-tag region. The 0-$b$-tag regions are used only for validation purposes.}
        \label{fig:hadhadABCDMethod}
    \end{figure}
    
    
The composition of these regions is shown in Appendix~\ref{subsec:appendix_bkg_multijetj_HadHad}. Table~\ref{tab:multijetSubtractionDTT} and Table~\ref{tab:multijetSubtractionSTT} show the relative subtraction of non-multijet-fake backgrounds in these regions.   

\begin{table}
\centering
\begin{tabular}{|c|c|c|c|}
\hline
 & 1 $b$-tag SS ID & 1 $b$-tag SS anti-ID & 2 $b$-tags OS anti-ID\\
\hline
1-prong & 12\% & 6.6\% & 54\%\\
3-prongs & 12\% & 6.8\% & 48\%\\
\hline
\end{tabular}
\caption{Relative subtraction of non-multijet-fake backgrounds in the DTT control regions used for the multijet estimation.}
\label{tab:multijetSubtractionDTT}
\end{table} 

\begin{table}
\centering
\begin{tabular}{|c|c|c|c|}
\hline
 & 1 $b$-tag SS ID & 1 $b$-tag SS anti-ID & 2 $b$-tags OS anti-ID\\
\hline
Leading \tauhad fails (1-prong) & 14\% & 3.8\% & 12\%\\
SubLeading \tauhad fails (1-prong) & 16\% & 10\% & 33\%\\
Leading \tauhad fails (3-prongs) & 20\% & 4.5\% & 18\%\\
SubLeading \tauhad fails (3-prongs) & 14\% & 12\% & 32\%\\
\hline
\end{tabular}
\caption{Relative subtraction of non-multijet-fake backgrounds in the STT control regions used for the multijet estimation.}
\label{tab:multijetSubtractionSTT}
\end{table} 
    
When only the \tauhad-ID of the $p_T$-sub-leading \tauhad\ candidate is inverted
for defining the anti-ID OS and SS CRs, a full multijet estimation is obtained.
Similarly, when only the \tauhad-ID of the $p_T$-leading \tauhad\ candidate is
inverted, another multijet estimation is obtained. To increase the number of
events that is used to model the multijet background, these two estimations are
averaged. The background templates that the fake-factors are applied to are
statistically-independent. In the STT category, due to low statistics, $FF$s are
measured inclusively in the \tauhad\ $p_T$ and $\eta$.

In the DTT category, it can be shown that the two sets of fake-factors, $FF_1$ (binned
in the $p_T$ of the sub-leading \tauhad, when only its ID is inverted) and
$FF_0$ (binned in the $p_T$ and $\eta$ of the leading \tauhad, when only its ID is
inverted), are statistically compatible (Appendix~\ref{subsec:appendix_bkg_multijetj_HadHad}: Figures~\ref{fig:app_bkg_HadHad_FFaveraging1P} and~\ref{fig:app_bkg_HadHad_FFaveraging3P}). For that reason, the fake-factors $FF_0$ and $FF_1$ are also
averaged, with taking into account the statistical significance they have in a
given $p_T$ bin\footnote{For example, $FF_0$ will not contribute to the
  inclusive fake-factor below 40~GeV, as there is a cut of 40~GeV applied to the
  leading \tauhad\ in the analysis.}. Thus, fake-factors in the DTT category,
are calculated as:

\begin{equation}
  FF_i(\pT \, \tauhad^i, \eta \, \tauhad^i, N_\text{prong} \, \tauhad^i, \dots)
  = \frac{ N_{\text{data}}(\text{loose} \, \tauhad^i)-N_{\text{non-multijet MC}}(\text{loose} \, \tauhad^i) }
  { N_{\text{data}}(\text{anti-}\tauhad^i) - N_{\text{non-multijet MC}}(\text{anti-}\tauhad^i) },
  \label{eq:HH:hadhad:multijet:FFi}
\end{equation}

where $\tauhad^i$ corresponds to the leading ($i = 0$) or subleading \tauhad ($i
= 1$) respectively. $N_{\text{data(non-multijet MC)}}$ is the number of \tauhad\
candidates in data (simulated non-multijet) events falling into the particular
$FF$ bin.

The two fake-factors, $FF_0$ and $FF_1$, are subsequently averaged using:

\begin{equation}  
  FF_\text{avg}(\pT \, \tauhad,\eta \, \tauhad, N_\text{prong} \, \tauhad, \dots) = \frac{N_{\text{data}}\text{(loose \tauhad)}-N_{\text{non-multijet MC}}\text{(loose \tauhad)}}{N_{\text{data}}\text{(anti-\tauhad)}-N_{\text{non-multijet MC}}\text{(anti-\tauhad)}},
  \label{eq:HH:hadhad:multijet:FFavg}
\end{equation}

where $N_{\text{data(non-multijet MC)}}$ in this case corresponds to the number
of \tauhad candidates in data (simulated non-multijet) events independently of
whether the \tauhadvis candidate is leading or subleading in \pT. There can be
up to two loose \tauhad\ per event in the numerator (if they are in the same
$FF$ bin), and up to one anti-\tauhad\ per event in the denominator. As already
mentioned, fake-factors defined in this way are equivalent to averaging $FF_0$
and $FF_1$, weighted by the relative number of events they contribute to a given
$p_T$ and $\eta$ bin. These fake-factors are applied to events with one anti-\tauhad, based
on its $p_T$ and $\eta$, regardless of whether this \tauhad\ is a $p_T$-leading or
sub-leading \tauhad\ in the event\footnote{The resulting multijet estimate is
  multiplied by $1/2$ effecively averaging the predictions from events where the
  leading \tauhad and the subleading \tauhad is failing identification.}. An
advantage of this approach compared to the previously-used approach with
2-dimensional fake-factor $p_T$-binning, used in the previous round of this analysis~\cite{HIGG-2016-16}, is that a finer binning can be used in
the new approach and it is easier to compare the compatibility of $FF$s between
the OS and SS regions when deriving systematic uncertainties. Nevertheless, the
two approaches give statistically compatible results, as shown in Appendix~\ref{subsec:appendix_bkg_multijetj_HadHad}: Figure~\ref{fig:app_bkg_HadHad_1Dvs2D}. Additionally, a very good agreement between the multijet estimate 
obtained when applying separately $FF_0$ and $FF_1$ to the corresponding 
events and the nominal estimate obtained when using $FF_{\mathrm{avg}}$ is 
shown in Appendix~\ref{subsec:appendix_bkg_multijetj_HadHad}: Figure~\ref{fig:app_bkg_HadHad_FFivsFFavg}.

As already mentioned, a set of transfer-factors is used to correct the
normalisation of the multijet prediction in the 2-$b$-tag region given that
fake-factors obtained in the 1-$b$-tag region are used. A transfer-factor is
defined as the ratio of the $p_T$/$\eta$-inclusive fake-factor in the 2-$b$-tag SS
region to the $p_T$/$\eta$-inclusive fake-factor in the 1-$b$-tag SS region,
inclusively derived for the STT and DTT trigger categories. Four
transfer-factors are defined, separately for 1- and 3-prong \tauhad, and
separately for the leading and the sub-leading \tauhad. The STT and DTT
fake-factors as well as the four transfer factors are given in
\Cref{fig:hadhad_fake_factors} and \Cref{fig:hadhad_transfer_factors} respectively.

\begin{figure}[h]
  \centering

  \includegraphics[width=.32\textwidth]{figures/bkg/hadhad_multijet/fake_factors/15_16/FF_1tag_TauPt_1P}
  \includegraphics[width=.32\textwidth]{figures/bkg/hadhad_multijet/fake_factors/17/FF_1tag_TauPt_1P}
  \includegraphics[width=.32\textwidth]{figures/bkg/hadhad_multijet/fake_factors/18/FF_1tag_TauPt_1P}

  \includegraphics[width=.32\textwidth]{figures/bkg/hadhad_multijet/fake_factors/15_16/FF_1tag_TauPt_3P}
  \includegraphics[width=.32\textwidth]{figures/bkg/hadhad_multijet/fake_factors/17/FF_1tag_TauPt_3P}
  \includegraphics[width=.32\textwidth]{figures/bkg/hadhad_multijet/fake_factors/18/FF_1tag_TauPt_3P}

  \includegraphics[width=.32\textwidth]{figures/bkg/hadhad_multijet/fake_factors/15_16/FF_STT_1tag}
  \includegraphics[width=.32\textwidth]{figures/bkg/hadhad_multijet/fake_factors/17/FF_STT_1tag}
  \includegraphics[width=.32\textwidth]{figures/bkg/hadhad_multijet/fake_factors/18/FF_STT_1tag}

  \caption{Fake factors for 1-prong DTT (top), 3-prong DTT (middle), and STT
    (bottom) for the data-taking periods 15-16 (left), 17 (center), and 18
    (right) for the di-Higgs \hadhad channel.}

  \label{fig:hadhad_fake_factors}
\end{figure}

\begin{figure}[h]
  \centering

  \includegraphics[width=.32\textwidth]{figures/bkg/hadhad_multijet/fake_factors/15_16/TF_1to2tag}
  \includegraphics[width=.32\textwidth]{figures/bkg/hadhad_multijet/fake_factors/17/TF_1to2tag}
  \includegraphics[width=.32\textwidth]{figures/bkg/hadhad_multijet/fake_factors/18/TF_1to2tag}

  \caption{Transfer factors for the data-taking periods 15-16 (left), 17
    (center), and 18 (right) for the di-Higgs \hadhad channel.}

  \label{fig:hadhad_transfer_factors}
\end{figure}

Validation plots are shown in Appendix~\ref{subsec:appendix_bkg_validation_hadhad}.

% Fake-factors used in the DTT category are binned in the \tahad\ \pT, while 

%\begin{itemize}
%\item “ID OS”: this is the SR, defined to contain events with two 'loose' RNN $\tau_{had}$ with opposite sign of the electric charge, as described in Section~\ref{subsec:selhh_hadhad};
%\item “ID SS”: control region where the two 'loose' RNN $\tau_{had}$ are required to have same sign electric charge instead of opposite sign;
%\item “anti-ID OS”: control region where instead of two 'loose' RNN $\tau_{had}$, one 'loose' $\tau$ and one anti-$\tau_{had}$ must be present, with anti$\tau$ defined as in Section~\ref{subsec:taus_antitaus}, and the two objects must have opposite sign electric charge;
%\item  “anti-ID SS”: control region defined as the previous region but the two objects are required to have same sign electric charge.
%\end{itemize}

%In each of the three control regions defined above the contributions from other background processes, corresponding to about 5\% of the number of data events in the region, are subtracted from the data using MC predictions. The fake-factors (FF) are then calculated from data in the ``SS'' regions as the ratio of the number of events in the ``ID SS'' region and the number of events in the ``anti-ID SS'' region:

%\begin{equation}
%FF = \frac{N^{ID}_{SS}}{N^{anti-ID}_{SS}}\,.
%\end{equation} 


%These FFs are then applied as event weights to the data events in the ``anti-ID OS'' region to obtain the estimation of QCD multi-jet events in the ``ID OS'' signal region:

%\begin{equation}
%N^{ID}_{OS}= N^{anti-ID}_{OS} \times FF\,,
%\end{equation}

%assuming that the ratio of events in the ``ID'' and ``anti-ID'' regions is the same for the ``SS'' and ``OS'' regions. In this way, multi-jet events in the opposite sign (OS) di-$\tau$ signal region are modelled with events from the anti-$\tau$ region, weighted by a fake-factor (FF). 
 
%The binning of the fake factors is dependent on the trigger that selected the
%event. For STT events the fake factor is binned in whether the anti-\tauhadvis
%is leading or subleading in \pT, and the decay mode of the \tauhadvis
%($N_\text{track}$). Due to low statistics in the STT category the fake factors
%are inclusive in \tauhadvis \pT. The \tauhadvis identification at the HLT is
%only applied to one of the two \tauhadvis candidates affecting the probability
%of jets faking \tauhadvis, motivating the binning in whether the leading /
%subleading \tauhadvis fails the identification.

%For DTT events, HLT \tauhadvis identification is applied to both \tauhadvis
%candidates. Therefore, the fake factors do not need to distinguish between cases
%where the leading and subleading \tauhadvis fails the loose identification. The
%fake factors for DTT events are parametrised in the \pT and decay mode of the
%the \tauhadvis candidate failing the identification requirement.

%Moreover, all fake factors are binned by data-taking period (${\text{2015-2016},
%  \text{2017}, \text{2018}}$) which takes into account the different triggers
%being used to select events used for the analysis.\todo{Chris: Might be changed soon}
% 
%The FFs are derived in the 1 $b$-tag region, which has larger data statistics compared to the 2 $b$-tags region.
%\textcolor{red}{To do: Chris will add info on transfer factor}
