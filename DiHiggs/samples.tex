\label{sec:DiHiggs:samples}



\section{Simulation of physics processes TODO: move this to theory section }
\label{sec:Simulation of physics processes}

A precise and reliable theoretical prediction is the key
to interpret and analyse the data recorded by the ATLAS experiment. 
It enbales the quantification of agreement between the data and the SM, 
and test of possible new physics beyond the SM. 
In this thesis, the signal and background processes are
simulated by Monte Carlo event generators.
The full simulation process is comprised of a few consecutive steps:
starting with the simulation of the hard-scattering process, 
followed by the simulation 
of the parton-showering and finally the simulation of 
the interaction of the particles with the detector and the 
response of the detector. 

The hard-scattering process happens when two partons which carry
a fraction of the protons collide inelasticly. 
The partons can be valence quarks, which are the quarks or anti-quarks
that determine the quantum numbers of the proton, or gluons 
which mediate the strong force, or sea quarks which are virtual 
quark-anti-quark pairs that are created and annihilated promptly. 
The hard-scatter process is typically characterised by large momentum 
transfer, 
and the probability of a parton carrying a given fraction of the total proton
momentum is described by the \textit{parton distribution function} (PDF).
In the MC simulation, the hard-scattering process is modelled by the
\textit{matrix element}, using leading order (LO) or next-to-leading oder (NLO)
Feynman diagrams. It can also be simulated at next-to-next-to-leading order (NNLO)
for better theoretical approximation. 

After the hard-scattering process, the parton will undergo the
showering process where it hadronises or radiates
further partons.
This process is simulated by dedicated algorithms. 
The hadronisation process happens mostly in low-energy regime,
which is non-perturbative and requires phenomenological modelling
exploiting specific hadronisation models. 
On the other hand, the radiation process happens at higher-energy, 
and it stops when the parton loses enough energy to reach the
confinement energy-scale.
The hard-scattering and showering processes of partons are 
described by the combination of the matrix element generator
and the parton shower algorithms. 

In addition to the hard-scattering and showering processes,
interaction can occur prior or after them, as referred to 
initial state radiation (ISR) or final state radiation (FSR).
The energy scale of these additional processes are typically a
few GeV, much smaller compared to the hard-scattering energy scale.
The radiated gluons or photons, together with other particles originating from
soft scattering interactions, are described as \textit{underlying event}.
% and they are characterised by a uniformly distributed underlying
% activity in the form of hadrons.

In addition, pileup events, as described in section~\ref{sec:LHC:pileup},
need to be simulated. 
They are separate collisions and are simulated 
separately from the hard-scattering
event. However, in the reconstruction of the event, 
they cannot be separated from the hard-scattering
event, therefore they are later overlaid on the hard-scattering
events.

A variety of MC generator programs were developed. 
Some of them are called multi-purpose generators and can
simulate a full-event on their own,
while some others are dedicated only to hard-scattering or
parton shower and need to be used combining with other generators. 
After simulation, 
simulated events are passed to \textsc{Geant~4}~\cite{Geant4,SOFT-2010-01}, 
a software package for simulating the interactions of particles with matter. 

Finally, events generators require \textit{tuning} to match data. 
The tuning parameters are based on phenomenological models and are
applied on hadronisation simulation and underlying event. 

\section{Data samples}
\label{sec:DiHiggs:data}

The data used in this search were collected at a 
centre-of-mass energy
of 13~TeV between 2015 and 2018.
For the FTAG calibration presented in Chapter~\ref{sec:FTAG}, 
the data sample was collected using a set of 
single-electron~\cite{TRIG-2018-05} and single-muon triggers~\cite{Aad:2020uyd}. 
Requirements on \pt\ over a range of 24--300~\GeV\ is applied 
to the single-electron triggers, with additional quality and 
isolation requirements depending on the \pt\ threshold and the 
data-taking period.
While for the single-muon triggers,
requirements on \pt\ over a range 20--26~\GeV\ are applied on the 
isolated muons, and a tigher cut of 50~\GeV\ is applied for muons 
without any isolation requirement. 

For the $HH$ searches presented in Chapter~\ref{sec:search for dihiggs},
single-lepton triggers and lepton-plus-\tauhad\ triggers are used. 
Details about these triggers are discussed in Section~\ref{sec:DiHiggs:selection}.

Events are selected for analysis only if they are of good quality and
if all the relevant detector components are known to be in operating conditions~\cite{DAPR-2018-01}.
The total integrated luminosity of the data, after meeting the good quality criteria,
is $139.0 \pm 2.4$~fb$^{-1}$~\cite{ATLAS-CONF-2019-021,ATLAS-Lumi2}.
The recorded events contain an average of 34 simultaneous inelastic $pp$ collisions per bunch-crossing.


\section{Simulated event samples}
\label{sec:DiHiggs:simulation}
As mentioned in section~\ref{sec:Simulation of physics processes},
the signal and background processes are modelled by Monte Carlo simulation. 
No signal sample is defined for the FTAG calibration. 
The signal targeted in the 
$HH$ searches
includes the SM-like non-resonant $HH$ production 
via ggF and VBF, 
and the BSM resonant $HH$ production. 
To simulate these processes, the full 
ATLAS detector simulation~\cite{SOFT-2010-01} (FS)
is applied, passing to the \textsc{Geant~4} package. 
The only exception is the resonant signal samples, 
where a fast simulation (AF2)~\cite{SOFT-2010-01} is used, 
that instead of fully simulating the response of the calorimeters,
it is replaced by pre-simulated shower to save computation time. 
For unstable hadrons i.e.\ 
$b$- and $c$-hadrons, the decay process is simulated 
by the \textsc{EvtGen~v1.6.0} package\cite{Lange:2001uf},
with the exception of the VBF non-resonant samples 
and samples generated by \textsc{Sherpa}~\cite{Bothmann:2019yzt}.
The resulting events were then
processed through the same reconstruction programs as the data.
To simulate the pileup effects, 
minimum bias events are overlaid on the simulation, 
exploting the \textsc{Pythia~8.186} generator~\cite{Sjostrand:2007gs}
for soft QCD processes using the A3~tune~\cite{ATL-PHYS-PUB-2016-017}.
The \textsc{NNPDF2.3LO}~\cite{Ball:2012cx} PDFs are used.
In addition, the Higgs boson mass was fixed to 125~GeV 
for all simulated samples that contain this particle.
The mass of the Higgs boson is assumed to be 125~GeV 
for all samples containing a SM Higgs boson.
This value is also used for calculating the 
single- and pair- production cross-sections 
of the Higgs boson, as well as the decay branching ratio of
the Higgs boson. 

A summary of the event samples used for the simulation of the signal and background
processes is shown in Table~\ref{tab:samples}.
\begin{landscape}
\newcommand{\hsp}{\hspace*{0.3cm}}% Discuss moving to source with EB
\begin{table}[tb!]
\begin{center}{\large


\resizebox{1.5\textwidth}{!}{\begin{tabular}{llllllll}
\toprule
\hsp Process & ME generator & ME QCD &  PDF & PS & Tuned parameters & Cross-section \hspace{2.5cm}\\
\midrule
\multicolumn{7}{l}{\BF{Signal}} \\
\hsp non-resonant $gg\to HH$ (ggF)   & \textsc{Powheg-Box~v2} & NLO & \textsc{PDF4LHC15} NLO & \textsc{Pythia~8.244} & A14 & NNLO FTApprox  \\
\hsp non-resonant $qq\to qqHH$ (VBF)   & \textsc{MadGraph}5\_aMC@NLO v2.7.3  & LO & NNPDF3.0NLO & \textsc{Pythia~8.244} & A14 & N3LO(QCD)  \\
\hsp resonant $gg\to X \to HH$ & \textsc{MadGraph}5\_aMC@NLO v2.6.1 & LO & NNPDF2.3LO & \textsc{Herwig~v7.1.3} & H7.1-Default & --  \\
\midrule
\multicolumn{7}{l}{\BF{Top-quark}} \\
\hsp $t\bar{t}$  & \textsc{Powheg-Box~v2} & NLO & NNPDF3.0NLO & \textsc{Pythia~8.230} & A14 & NNLO+NNLL \\
\hsp single top($t$-, $s$-, $Wt$-channels) & \textsc{Powheg-Box~v2} & NLO & NNPDF3.0NLO & \textsc{Pythia~8.230} & A14 & NLO \\
% \hsp $s$-channel & \textsc{Powheg-Box~v2} & NLO & NNPDF3.0NLO & \textsc{Pythia~8.230} & A14 & NLO \\
% \hsp $Wt$        & \textsc{Powheg-Box~v2} & NLO & NNPDF3.0NLO & \textsc{Pythia~8.230} & A14 & NLO \\
\hsp $t\bar{t}Z$ & \textsc{Sherpa~2.2.1}  & NLO & NNPDF3.0NNLO &  \textsc{Sherpa~2.2.1} & Default & NLO  \\
\hsp $t\bar{t}W$ & \textsc{Sherpa~2.2.8}  & NLO & NNPDF3.0NNLO &  \textsc{Sherpa~2.2.8} & Default & NLO  \\
\midrule
\multicolumn{7}{l}{\BF{Single Higgs boson}} \\
\hsp ggF         & \textsc{Powheg-Box~v2} & NNLO & NNPDF3.0NLO &  \textsc{Pythia~8.212} & AZNLO & N3LO(QCD)+NLO(EW)  \\
\hsp VBF         & \textsc{Powheg-Box~v2} & NLO  & NNPDF3.0NLO &  \textsc{Pythia~8.212} & AZNLO & NNLO(QCD)+NLO(EW)  \\
\hsp $qq\to WH, ZH$  & \textsc{Powheg-Box~v2} & NLO  & NNPDF3.0NLO &  \textsc{Pythia~8.212} & AZNLO & NNLO(QCD)+NLO(EW)  \\
% \hsp $qq\to ZH$  & \textsc{Powheg-Box~v2} & NLO  & NNPDF3.0NLO &  \textsc{Pythia~8.212} & AZNLO & NNLO(QCD)+NLO(EW)$^{(\dagger)}$  \\
\hsp $gg\to ZH$  & \textsc{Powheg-Box~v2} & NLO  & NNPDF3.0NLO &  \textsc{Pythia~8.212} & AZNLO & NLO+NLL \\
\hsp $t\bar{t}H$ & \textsc{Powheg-Box~v2} & NLO  & NNPDF3.0NLO &  \textsc{Pythia~8.230} & A14   & NLO  \\
\midrule
\multicolumn{7}{l}{\BF{Vector boson + jets}} \\
\hsp $W/Z+$jets & \textsc{Sherpa~2.2.1} & NLO ($\leq 2$ jets), LO  (3,4 jets)  & NNPDF3.0NNLO &  \textsc{Sherpa~2.2.1} & Default & NNLO  \\
\midrule
\multicolumn{7}{l}{\BF{Diboson}} \\
\hsp $WW,WZ,ZZ$ & \textsc{Sherpa~2.2.1} & NLO ($\leq 1$ jet), LO (2,3 jets)   & NNPDF3.0NNLO &  \textsc{Sherpa~2.2.1} & Default & NLO  \\
\bottomrule
\end{tabular}}}
\end{center}
\caption
{The generators used for the simulation of the signal and background
processes. The order of the cross-section calculation refers
to the expansion in the strong coupling constant ($\alpha_\text{S}$).
The acronyms ME, PS and UE are used for matrix element, parton shower and underlying event, respectively.
The terms ggF, VBF refer to gluon-gluon fusion and vector-boson fusion respectively. 
The cross-section of the resonant production is not shown as a dummy cross-section of 1 pb was chosen.
Reproduced from Ref.~\cite{dihiggs-conf}.
\protect\label{tab:samples}}
\end{table}
% Discuss moving to source with EB
\end{landscape}
\subsection{Simulated signal samples}

In the $HH$ searches, contributions from both the ggF and VBF to the 
SM non-resonant $HH$ signal production are included, 
each simulated with different generators and PDFs.
The expansion order of Feynman diagrams are also different. 
While for the resonant $HH$ signal, only the ggF contribution 
is considered.
It was simulated for 20 values of the resonance mass, $m_{X}$,
between 251~GeV and 1600~GeV
(251, 260, 280, 300, 325, 350, 375,
 400, 450, 500, 550, 600, 700, 800, 
 900, 1000, 1100, 1200, 1400 and 1600).

The ggF events were generated with 
the \textsc{Powheg-Box~v2} generator~\cite{Alioli:2010xd}
at next-to-leading order (NLO) with finite top-quark mass,
using the \textsc{PDF4LHC15} NLO PDF set~\cite{Butterworth:2015oua}.
Parton showers and hadronisation were interfaced to 
\textsc{Pythia~8.244}~\cite{Sjostrand:2007gs}
with the A14 set of tuned parameters~\cite{ATL-PHYS-PUB-2014-021,ATLAS:2012uec} 
and the \textsc{NNPDF2.3LO} PDF set.


TODO: move this paragraph to theory 
The cross-section of the ggF non-resonant $HH$ production is
calculated at next-to-next-to-leading order (NNLO) FTApprox~\cite{Grazzini:2018bsd},
taking into account the finite top-quark mass assumption. The cross-section is \linebreak
\mbox{$31.05^{+2.2\%}_{-5.0\%}\text{(scale)}\pm 2.1\%(\alpha_\text{S})\pm 2.1\%\text{(PDF)}\pm 2.6\%(\text{m}_\text{top})$ fb}
at $\sqrt{s}=13$ TeV and $\text{m}_{H}=125$ GeV~\cite{dihiggs-twiki}.
The scale uncertainty is due to 
the finite order of quantum chromodynamics (QCD) calculations,
the $\alpha_\text{s}$ and PDF terms 
account for the uncertainties on the strong coupling constant 
and parton distribution functions respectively, and the 
$m_\text{top}$ uncertainty is related to the top-quark mass scheme.
The normalisation of this process is set to the production cross-section 
times the \bbtautau\ branching ratio (BR), \linebreak
\mbox{$\sigma_{ggF} \times BR_{\bbtautau}  = \SI{31.05}{fb} \times 0.0730  =  \SI{2.268}{fb}$}.  


On the other hand, the VBF events generated using the
\textsc{MadGraph}5\_aMC@NLO v2.7.3~\cite{Alwall:2014hca} generator at LO 
with the \textsc{NNPDF3.0NLO}~\cite{Ball:2014uwa} PDF set.
Parton showering and hadronisation were simulated using \textsc{Pythia~8.244}
with the A14 tune and the \textsc{NNPDF2.3LO} PDF set.
To simulate the decays of the $b$- and $c$-hadrons,
the \textsc{EvtGen} v1.7.0 program was used.

TODO: move this paragraph to theory
The cross-section of the VBF non-resonant $HH$ production calculated at
next-to-next-to-next-to-leading order (N3LO) in QCD in the limit 
in which there is no partonic exchange between the two protons~\cite{Dreyer:2018qbw} is
$1.726^{+0.03\%}_{-0.04\%}\text{(scale)}\pm 2.1\%(\text{PDF}+\alpha_\text{S})$ fb~\cite{dihiggs-twiki}.
This process is normalised to the cross-section times the \bbtautau\ branching ratio, \linebreak[2]
\mbox{$\sigma_{VBF} \times BR_{\bbtautau}  = \SI{1.726}{fb} \times 0.07306  =  \SI{0.1261}{fb}$.}  
Other non-resonant $HH$ production modes are not considered as their contributions
to the analysis sensitivity are expected to be negligible.



Finally, the resonant signal  
of a heavy spin-0 narrow width resonance via ggF production 
was simulated with the \textsc{MadGraph}5\_aMC@NLO v2.6.1 
generator using the \textsc{NNPDF2.3LO} PDF
set at LO.
The parton shower and hadronisation were simulated to
\textsc{Herwig~7.1.3}~\cite{Bahr:2008pv,Bellm:2015jjp},
using the H7.1-Default tune~\cite{Gieseke:2012ft}
and the \textsc{NNPDF2.3LO} PDF set.


TODO:move this paragraph to theory
The normalisation of these resonant signal samples in the analysis is set to \linebreak[2]
\mbox{$\sigma \times BR  = \SI{1}{pb} \times BR (bb \tau \tau)  =  0.07306 \SI{1}{pb}$.} 
A dummy cross-section of 1 pb is chosen for these samples 
to make combination with other decay channels easier, 
and to ease scaling of the signal 
for the calculation of the cross-section limits 
(more details in Section \ref{sec:DiHiggs:results}).

\subsection{Non-resonant signal reweighting and combination}
In the BSM scenarios, the non-resonant di-Higgs production is sensitive 
to the self-coupling constant and other possible anomalous coupling,
as described in section~\ref{sec:Theory:kl} 
(TODO: ref back to theory section about klambda dependence).
In this thesis, a reweighting method is used to evaluate
the non-resonant di-Higgs production with a range of possible 
values of self-coupling modifier, $k_\lambda$, and 7 benchmark models (BM)
of a set of five HEFT couplings. 
% represents the deviation of the Higgs boson self-coupling from the SM expected value. 
% The non-resonant $HH$ signal samples are generated for 
% only a limited number of $\kappa_\lambda$ points due to heavy computational cost. 
\paragraph{$k_\lambda$ Reweighting}
For the \ggH non-resonant $HH$ production, MC samples are generated at \kl=1 and 10, 
while for \VBFH production, 
MC samples are generated at \kl=0, 1, 2 and 10. 
A sample combination technique is used to model the signal hypothesis at different \kl values.
For the \ggH di-Higgs production, 
a reweighting method described in Ref.~\cite{ATL-PHYS-PUB-2019-007} 
is used to obtain predictions at different $\kl$ values 
in the range $\kl \in \left[-30, 30 \right] $ 
in increments of 0.2 based on a linear combination of 
% truth level  %% how to define truth level? 
generator samples at $\kl=0, 1$ and 20. 
The remaining $\kl = 10$ sample is used to validate the method.
%  at truth level. 
For each \kl value, a set of weights $w(m_{HH},\kl)$ is evaluated 
by dividing the binned $m_{HH}$ distribution of the \kl target sample 
by the SM distribution. 
They can be used to reweight the SM non-resonant sample to any \kl value,
which is performed at analysis level--after reconstruction and the selection steps
defined in section~\ref{sec:DiHiggs:selection}. 
Given the assumption that the kinematic of the \ggH events 
and their acceptance depend only on $m_{HH}$ variable, 
using the weights $w(m_{HH},\kl)$ the reweighted sample describes correctly any kinematic distribution of a given target \kl value. 
Good closure is found between the distributions obtained from \kl generated and reweighted, 
as shown in section~\ref{sec:DiHiggs:systemmatics} TODO: add reference to systematics section.
% To evaluate the sensitivity of the analyses to \kl, 
% the SM \ggH sample is processed through the analyses and then it is reweighted to different values of \kl. 
% The \bbyy analysis considers only the signal yield variation 
% as it was checked that \kl variation in the $m_{\gamma\gamma}$ shape has negligible impact on the final limits, 
% while the \bbtautau analysis takes into account 
% the MVA output shape changes due to the modification of \kl in addition to the signal yield variation. 

For the \VBFH di-Higgs production, the above reweighting procedure is not valid, 
because the kinematic of the events can not be defined 
just using a single variable such as $m_{HH}$. 
Instead, three fully-reconstructed MC samples with $\kl = 1,2,10$ are used.
The event distributions and the multivariate algorithm output 
(more details in section~\ref{sec:DiHiggs:MVA}) 
for any \kl value are obtained from the linear combination 
of the corresponding distributions 
of the three samples at analysis level. 
As defined in section~TODO: ref back to theory chapter,
the full cross-section for the VBF $HH$ production involves 
three diagrams, and expanding the absolute squared of
the amplitude yileds six terms:
\[
\sigma = \kappa^2_V \kappa^2_\lambda a_1 + \kappa_V^4 a_2
+ \kappa_{2V}^2 a_3 + \kappa_V^3 \kappa_\lambda a_4
+ \kappa_V\kappa_\lambda\kappa_{2V} a_5 
+ \kappa_V^2 \kappa_{2V}a_6 . \]
In the case of \kl scan, this formula is reduced to 
\[
\sigma = \kappa^2_\lambda a_1 + \kappa_\lambda a_2 + a_3 .
\]
Using the basis of 
$\kl = 1,2,10$,
the linear coefficients for combining the 
three samples defined are then given by:
\begin{eqnarray*}
    \sigma (\kl) =\left(\frac{\kl^{2}}{9} - \frac{4\kl}{3} + \frac{20}{9} \right) \times \sigma (1) + \left(-\frac{\kl^{2}}{8} + \frac{11\kl}{8} - \frac{5}{4} \right) \times \sigma (2)
\end{eqnarray*}
\begin{eqnarray}
      + \left(\frac{\kl^{2}}{72} - \frac{\kl}{24} + \frac{1}{36} \right) \times \sigma (10) 
    \label{eqn:vbf-linear-comb}
\end{eqnarray}
 
\paragraph{HEFT Benchmarks reweighting}
The reweighting is only applied to the ggF non-resonant production. 
A similar approach is used: for each BM, a set of weights $w(m_{HH},BM)$ is evaluated 
by dividing the binned $m_{HH}$ distribution of the BM sample by the SM distribution;
these weights can then be used to 
reweight the SM non-resonant sample to any BM signals as 
defined in section~TODO: ref back to theory chapter. 

TODO: move this paragraph and table to theory
The assumed coulpings values and the corresponding cross-sections
of these BMs are shown in Table~\ref{tab:HEFT values}.
These BM points are defined to represent the different possible characteristic shapes
in $m_{HH}$ distribution, as described in Ref.~\cite{HEFT-BM}.
\begin{table}[tb!]
    \begin{center}{\large    
    \resizebox{0.6\textwidth}{!}{\begin{tabular}{llllllll}
    \toprule
    Benchmark & $c_t$ & $c_{hhh}$ &  $c_{tt}$ & $c_{ggh}$ & $c_{gghh}$ & Cross-sections (fb)\\
    \midrule
    SM  &  1  & 1 & 0 & 0 & 0 & 31.05  \\ 
    BM1 &  0.94 &3.94 &-1/3 &0.5 &1/3 &181.40 \\
    BM2 &  0.61 &6.84 &1/3 &0.0 &-1/3 &135.24 \\
    BM3 &  1.05 &2.21 &-1/3 &0.5 &0.5 &108.90 \\
    BM4 &  0.61 &2.79 &1/3 &-0.5 &1/6 &50.45 \\
    BM5 &  1.17 &3.95 &-1/3 &1/6 &-0.5 &117.04 \\
    BM6 &  0.83 &5.68 &1/3 &-0.5 &1/3 &144.91 \\
    BM7 &  0.94 &-0.10 &1 &1/6 &-1/6 &97.96 \\
    \midrule
    \bottomrule
    \end{tabular}}}
    \end{center}
    \caption
    {The proposed coupling values and the corresponding cross-sections of each BMs. 
    The physical meaning of each coupling is described in section~\ref{sec:Theory} TODO: add reference to theory section.
    \label{tab:HEFT values}}
    \end{table}
    


\subsection{Simulated background samples}
\label{sec:MC samples}
The major background processes for the 
$HH$ searches are the \ttbar, single-top, 
boson produced in association with jets and jets faking a 
\tauhad\ (more details in section~\ref{sec:DiHiggs:lephadfake}). 
Minor background processes include the Drell-Yan processes, 
processes with dibon final states and single-Higgs processes.
These background processes, except for the jets faking a 
\tauhad\ background, are modelled by the full ATLAS 
detector simulation. 
For the FTAG calibration, the events used for calibration
mostly originate from the \ttbar\ processes. Other minor 
background includes the single-top, diboson,  
production of \ttbar\ in association with a boson 
and boson produced in association with jets. 
Likewise, all processes are passed through the 
full simulation. The generators, PDFs,
expansion order and tune used in the $HH$ searches
and the FTAG calibration for the samples in common 
are in general identical, unless specified. 

The $t\bar{t}$ production and the single top-quark events 
in the $Wt$-, $s$- and $t$-channels
were simulated by the \textsc{Powheg-Box v2} generator~
\cite{Frixione:2007nw,Nason:2004rx,Frixione:2007vw,Alioli:2010xd}
together with the \textsc{NNPDF3.0NLO} PDF set~\cite{Ball:2014uwa}.
The showering, hadronisation and underlying event 
are modeled by \textsc{Pythia~8.230},
with parameters set according to the A14 tune~\cite{ATL-PHYS-PUB-2017-007}
and using the \textsc{NNPDF2.3LO} PDF set.
The top-quark mass was set to 172.5~GeV,
with top-quark spin correlations preserved.
To achieve the better accuracy, 
the $t\bar{t}$ production cross-section is calculated
at next-to-next-to-leading order 
and next-to-next-to-leading-logarithm 
(NNLO+NNLL)~\cite{Czakon:2011xx}.
While for the single-top process, 
the cross-sections of are calculated at
NLO~\cite{stop_sch_Xsec,stop_tch_Xsec,Kidonakis:2010ux}.
The $t\bar{t}$-$Wt$ interference is removed
using the diagram removal scheme~\cite{Frixione:2008yi}
(more details in Section~\ref{} TODO: add reference to the systematics for singletop ).

The production of bosons in association with jets 
($W/Z$+jets) are simulated by 
\textsc{Sherpa}~2.2.1 generator~\cite{Bothmann:2019yzt}
using the NNPDF3.0NNLO~\cite{Ball:2014uwa} PDF set.
The tuning used for parton shower is developed by the 
\textsc{Sherpa} authors. 
The matrix elements are simulated for up to two partons at
NLO and up to four partons at LO, calculated with the 
\textsc{Comix}~\cite{Gleisberg:2008fv} and 
\textsc{OpenLoops}~\cite{Cascioli:2011va} libraries.
The samples are normalised to NNLO prediction~\cite{Anastasiou:2003ds}.

Similar settings apply to the diboson ($WW$, $WZ$ and $ZZ$) events.
\textsc{Sherpa}~2.2.1 generator is used to simulate these processes
for the $HH$ searches, while \textsc{Sherpa}~2.2.1 and \textsc{Sherpa}~2.2.2
are used for the FTAG calibration depending on the process.
For both the $HH$ searches and FTAG calirabtion, 
diboson samples are simulated using
matrix elements at NLO accuracy in QCD for up to one additional parton
and at LO accuracy for up to three additional parton emissions.
The cross-section is calculated at NLO accuracy, while
the rest of the settings remains the same as the ones for the boson+jets background.

In addition, the events where
a vector boson is produced in association with \ttbar\
($t\bar tZ$ and $t\bar tW$) are generated differently
between the $HH$ searches and the FTAG calibration.
In the former,
\textsc{Sherpa}~2.2.1 (2.2.8) is used to simulate the
$t\bar tZ$ ($t\bar tW$) production at NLO.
The cross-sections are calculated at NLO accuracy and 
the rest remains the same as the settings for the boson+jets. 
While for the FTAG calibration,
the $t\bar tZ$ and $t\bar tW$ events are modelled using the
\mgamc~v2.3.3~\cite{Alwall:2014hca} generator at NLO with the
\nnpdfnlo~\cite{Ball:2014uwa} parton distribution function~(PDF).
The events are interfaced to \pythia.210~\cite{Sjostrand:2014zea}~
using the A14 tune~\cite{ATL-PHYS-PUB-2014-021} and the
\nnpdftwo~\cite{Ball:2014uwa} PDF set. The decays of bottom and charm
hadrons are simulated using the \evtgen\ v1.2.0 program~\cite{Lange:2001uf}.


In the $HH$ searches presented in this thesis, 
the SM single Higgs boson production 
is considered as part of the background.
It has played a non-negligible role as a background 
to the analysis, especially in the non-resonant search.
This due to the similar kinematics between the 
single- and double- Higgs production.
The simulated events of the single Higgs boson production, 
in various processes, were generated using 
\textsc{Powheg-Box v2} generator and NNPDF3.0NNLO PDF set.


The ggF single Higgs production is interfaced to the 
Powheg NNLOPS program~\cite{Hamilton:2013fea,Hamilton:2015nsa}
at NNLO accuracy. 
The cross-section is calculated at Next-to-Next-to-Next-to-Leading
order (N3LO) for the QCD processes and NLO for the electroweak expansion
~\cite{deFlorian:2016spz,Anastasiou:2016cez,Anastasiou:2015ema,Dulat:2018rbf,Actis:2008ug}.
On the other hand, 
the VBF single Higgs production events are 
interfaced to \textsc{Pythia~8.212} using the 
\textsc{CTEQ6L1} PDF set~\cite{CTEQ6L1},
together with the AZNLO tune~\cite{AZNLOtune}.
The cross-section of the VBF process is 
calculated at NNLO for QCD processes and 
NLO for electroweak expansion
~\cite{deFlorian:2016spz,Ciccolini:2007jr,Ciccolini:2007ec,Bolzoni:2010xr}.


The $qq \rightarrow WH$, $qq \rightarrow ZH$ and $gg \rightarrow ZH$ simulated events
were interfaced to \textsc{Pythia~8.212} for parton shower
and hadronisation using the AZNLO tune together with the
\textsc{CTEQ6L1} PDF set.
The cross-sections are set to the NNLO(QCD)+NLO(EW) calculations for
$qq \rightarrow WH$ and 
$qq \rightarrow ZH$~\cite{Ciccolini:2003jy,Brein:2003wg,Ferrera:2011bk,Brein:2011vx,Ferrera:2013yga,Ferrera:2014lca,Campbell:2016jau},
and to the next-to-leading order and next-to-leading-logarithm (NLO+NLL) in QCD for
$gg \rightarrow ZH$~\cite{Altenkamp:2012sx,Hespel:2015zea,ggzhnll,Harlander:2013mla,Brein:2012ne}.


For processes where single Higgs produced in association with a vector boson, 
i.e.\ the $qq \rightarrow ZH$, $gg \rightarrow ZH$ and $qq \rightarrow WH$, 
processes, \textsc{Pythia~8.212} is used to simulate the parton shower
and hadronisation using the AZNLO tune, together with the 
\textsc{CTEQ6L1} PDF set.
For the $qq \rightarrow WH$ and the $qq \rightarrow ZH$ sample, 
the cross-section is calculated at NNLO for QCD processes and 
NLO for electroweak expansion
~\cite{Ciccolini:2003jy,Brein:2003wg,Ferrera:2011bk,Brein:2011vx,
Ferrera:2013yga,Ferrera:2014lca,Campbell:2016jau}; 
while for $gg \rightarrow ZH$ the cross-section is calculated 
at NLO for QCD and next-to-leading lograithmic (NLL) for electroweak expansion
~\cite{Altenkamp:2012sx,Hespel:2015zea,ggzhnll,Harlander:2013mla,Brein:2012ne}. 

Finally, the simulated events with a single Higgs produced in association with 
a pair of top-quarks are interfaced to \textsc{Pythia~8.230}.
The parton shower and hadronisation are set to the 
A14 tune, using the \textsc{NNPDF2.3LO} PDF set.
The cross-section is calculated at NLO accuracy~\cite{deFlorian:2016spz}.

In the $HH$ searches and FTAG calibration presented in this thesis, 
additional samples were produced with alternative generators or settings,
in order to estimate systematic uncertainties in the event modelling.
These alternative will be defined once they are introduced in the content. 
