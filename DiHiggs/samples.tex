\section{Data and Monte Carlo Simulation}
\subsection{Data samples}
\label{sec:DiHiggs:data}

The data used in this search were collected at a centreof-mass energy
of 13~TeV between 2015 and 2018, using triggers to 
select events with at least one electron or one muon
and at least one $\tau_\text{had-vis}$.
% at least one lepton and one $\tau_\text{had-vis}$,
% or at least two $\tau_\text{had-vis}$.
Details about these triggers are discussed in Section~\ref{sec:DiHiggs:triggers}.
Events are selected for analysis only if they are of good quality and
if all the relevant detector components are known to be in good operating conditions~\cite{DAPR-2018-01}.
The total integrated luminosity of the data, after meeting the good quality criteria,
is $139.0 \pm 2.4$~fb$^{-1}$~\cite{DAPR-2013-01,ATLAS-Lumi2}.
The recorded events contain an average of 34 simultaneous inelastic $pp$ collisions per bunch-crossing.

\subsection{Simulated event samples}
\label{sec:DiHiggs:simulation}
Monte Carlo (MC) simulated events are used 
to model the SM background production,
the SM-like non-resonant $HH$ signal production, 
and the BSM resonant $HH$ signal production.
They are passed through the full ATLAS detector simulation~\cite{SOFT-2010-01}
based on \textsc{Geant~4}~\cite{Agostinelli:2002hh}, 
except for the BSM resonant $HH$ signal samples
that are passed through a fast simulation, 
for which the response of the calorimeters is parameterised
rather than fully simulated.
The effects of pileup are modelled by overlaying minimum bias events, 
simulated using the soft quantum chromodynamics (QCD) processes
of \textsc{Pythia~8.186}~\cite{Sjostrand:2007gs} with the A3~\cite{ATL-PHYS-PUB-2016-017}
set of tuned parameters and \textsc{NNPDF2.3LO}~\cite{Ball:2012cx} parton distribution functions (PDFs).
The \textsc{EvtGen~v1.6.0} program~\cite{Lange:2001uf}
is used to describe the decays of bottom and charm hadrons
for all samples of simulated events, except for those generated using
\textsc{Sherpa}~\cite{Bothmann:2019yzt} and the VBF non-resonant $HH$ sample.
The resulting events were then processed through 
the same reconstruction programs as the data.
%%% The Higgs boson mass was fixed to 125~GeV for all simulated samples that contain this particle.
For all samples containing a SM Higgs boson, 
its mass was fixed to 125 GeV.
The same mass value is used for the calculation of the Higgs boson decay branching ratios
and for the calculation of the single Higgs boson and SM non-resonant $HH$ production
cross-sections.
Unless specified, the order of the cross-section calculation
refers to the expansion in the strong coupling constant ($\alpha_\text{s}$).
A summary of the event samples used for the simulation of the signal and background
processes is shown in Table~\ref{tab:samples}.
\newcommand{\hsp}{\hspace*{0.3cm}}% Discuss moving to source with EB
\begin{table}[tb!]
\begin{center}{\small


\resizebox{\textwidth}{!}{\begin{tabular}{llllllll}
\toprule
\hsp Process & ME generator & ME QCD & ME PDF & PS and & UE model & Cross-section \hspace{2.5cm}\\
& & & hadronisation & tune & order\\
\midrule
\multicolumn{7}{l}{\BF{Signal}} \\
\hsp non-resonant $gg\to HH$ (ggF)   & \textsc{Powheg-Box~v2} & NLO & \textsc{PDF4LHC15} NLO & \textsc{Pythia~8.244} & A14 & NNLO FTApprox  \\
%
\hsp non-resonant $qq\to qqHH$ (VBF)   & \textsc{MadGraph}5\_aMC@NLO & LO & NNPDF3.0NLO & \textsc{Pythia~8.244} & A14 & N3LO(QCD)  \\
                                       & v2.7.3                      &   &             &                       &     &            \\
%
\hsp resonant $gg\to X \to HH$ & \textsc{MadGraph}5\_aMC@NLO & LO & NNPDF2.3LO & \textsc{Herwig~v7.1.3} & H7.1-Default & --  \\
                               & v2.6.1                      &    &            &                        &              &     \\
%
% \hsp scalar LQ pair & \textsc{MadGraph5\_aMC@NLO} & NNPDF3.0NLO &\textsc{Pythia~8.230}~\cite{PYTHIA82} & A14~\cite{ATL-PHYS-PUB-2014-021} & NNLO+NNLL  \\
%
\midrule
\multicolumn{7}{l}{\BF{Top-quark}} \\
\hsp $t\bar{t}$  & \textsc{Powheg-Box~v2} & NLO & NNPDF3.0NLO & \textsc{Pythia~8.230} & A14 & NNLO+NNLL \\
\hsp $t$-channel & \textsc{Powheg-Box~v2} & NLO & NNPDF3.0NLO & \textsc{Pythia~8.230} & A14 & NLO \\
\hsp $s$-channel & \textsc{Powheg-Box~v2} & NLO & NNPDF3.0NLO & \textsc{Pythia~8.230} & A14 & NLO \\
\hsp $Wt$        & \textsc{Powheg-Box~v2} & NLO & NNPDF3.0NLO & \textsc{Pythia~8.230} & A14 & NLO \\
\hsp $t\bar{t}Z$ & \textsc{Sherpa~2.2.1}  & NLO & NNPDF3.0NNLO &  \textsc{Sherpa~2.2.1} & Default & NLO$^{(\ddagger)}$  \\
\hsp $t\bar{t}W$ & \textsc{Sherpa~2.2.8}  & NLO & NNPDF3.0NNLO &  \textsc{Sherpa~2.2.8} & Default & NLO$^{(\ddagger)}$  \\
\midrule
\multicolumn{7}{l}{\BF{Vector boson + jets}} \\
\hsp $W/Z+$jets & \textsc{Sherpa~2.2.1} & NLO ($\leq 2$ jets) & NNPDF3.0NNLO &  \textsc{Sherpa~2.2.1} & Default & NNLO  \\
                &                       & LO  (3,4 jets)     &              &                        &                 \\
%%% \hsp $Z+$jets & \textsc{Sherpa~2.2.1} &   & NNPDF3.0NNLO &  \textsc{Sherpa~2.2.1} & Default & NNLO  \\
\midrule
\multicolumn{7}{l}{\BF{Diboson}} \\
\hsp $WW,WZ,ZZ$ & \textsc{Sherpa~2.2.1} & NLO ($\leq 1$ jet)  & NNPDF3.0NNLO &  \textsc{Sherpa~2.2.1} & Default & NLO$^{(\ddagger)}$  \\
                &                       & LO (2,3 jets)       &              &                        &         &                     \\
%\hsp $WW$ & \textsc{Sherpa~2.2.1} &   & NNPDF3.0NNLO &  \textsc{Sherpa~2.2.1} & Default & NLO$^{(\ddagger)}$  \\
%\hsp $WZ$ & \textsc{Sherpa~2.2.1} &   & NNPDF3.0NNLO &  \textsc{Sherpa~2.2.1} & Default & NLO$^{(\ddagger)}$  \\
%\hsp $ZZ$ & \textsc{Sherpa~2.2.1} &   & NNPDF3.0NNLO &  \textsc{Sherpa~2.2.1} & Default & NLO$^{(\ddagger)}$  \\
\midrule
\multicolumn{7}{l}{\BF{Single Higgs boson}} \\
\hsp ggF         & \textsc{Powheg-Box~v2} & NNLO & NNPDF3.0NLO &  \textsc{Pythia~8.212} & AZNLO & N3LO(QCD)+NLO(EW)  \\
\hsp VBF         & \textsc{Powheg-Box~v2} & NLO  & NNPDF3.0NLO &  \textsc{Pythia~8.212} & AZNLO & NNLO(QCD)+NLO(EW)  \\
\hsp $qq\to WH$  & \textsc{Powheg-Box~v2} & NLO  & NNPDF3.0NLO &  \textsc{Pythia~8.212} & AZNLO & NNLO(QCD)+NLO(EW)  \\
\hsp $qq\to ZH$  & \textsc{Powheg-Box~v2} & NLO  & NNPDF3.0NLO &  \textsc{Pythia~8.212} & AZNLO & NNLO(QCD)+NLO(EW)$^{(\dagger)}$  \\
\hsp $gg\to ZH$  & \textsc{Powheg-Box~v2} & NLO  & NNPDF3.0NLO &  \textsc{Pythia~8.212} & AZNLO & NLO+NLL \\
\hsp $t\bar{t}H$ & \textsc{Powheg-Box~v2} & NLO  & NNPDF3.0NLO &  \textsc{Pythia~8.230} & A14   & NLO  \\
\bottomrule
\end{tabular}}}
\end{center}
\caption
{The generators used for the simulation of the signal and background
processes. If not specified, the order of the cross-section calculation refers
to the expansion in the strong coupling constant ($\alpha_\text{S}$).
The acronyms ME, PS and UE are used for matrix element, parton shower and underlying event, respectively.
\protect\label{tab:samples}}
\end{table}
% Discuss moving to source with EB
\subsubsection{Simulated signal samples}
Simulated SM non-resonant $HH$ signal production includes the contributions
from the ggF and VBF processes.
%%The simulated ggF events were generated at next-to-leading order (NLO) with
%%the \textsc{Powheg-Box~v2} generator~\cite{Alioli:2010xd} with full NLO corrections and with finite top-quark mass,
%%and using the \textsc{PDF4LHC15}~\cite{Butterworth:2015oua} PDF set.
The simulated ggF events were generated with the \textsc{Powheg-Box~v2} generator~\cite{Alioli:2010xd}
at next-to-leading order (NLO) with finite top-quark mass,
and using the \textsc{PDF4LHC15} NLO PDF set~\cite{Butterworth:2015oua}.
Parton showers and hadronisation were simulated using \textsc{Pythia~8.244}~\cite{Sjostrand:2007gs}
with the A14 set of tuned parameters~\cite{ATL-PHYS-PUB-2014-021,ATLAS:2012uec} and the \textsc{NNPDF2.3LO} PDF set.
%%The mass of the Higgs boson was fixed at 125~GeV for this simulated signal sample and other
%%samples used in the analyses reported here.
%%%%%%
%The cross-section of the ggF non-resonant $HH$ production,
%calculated at next-to-next-to-leading order (NNLO) FTApprox~\cite{Grazzini:2018bsd},
%is $31.1^{+2.2\%}_{-5.0\%}\text{(scale)}\pm 2.1\%(\alpha_\text{S})\pm 2.1\%\text{(PDF)}\pm 2.6\%(\text{m}_\text{top})$ fb
%at $\sqrt{s}=13$ TeV and $\text{m}_{H}=125$ GeV.
The cross-section of the ggF non-resonant $HH$ production 
calculated at next-to-next-to-leading order (NNLO) FTApprox~\cite{Grazzini:2018bsd} 
(taking account of the finite top-quark mass) is
$31.1^{+2.2\%}_{-5.0\%}\text{(scale)}\pm 2.1\%(\alpha_\text{S})\pm 2.1\%\text{(PDF)}\pm 2.6\%(\text{m}_\text{top})$ fb
at $\sqrt{s}=13$ TeV and $\text{m}_{H}=125$ GeV.

% The normalisation of this process in the analysis is set to the production cross-section 
% times the $bb\tau\tau$ BR, 
% $\sigma_{ggF} \times BR (bb \tau\tau)  = \SI{31.05}{fb} \times 0.0730562561  =  \SI{2.2683967}{fb}$.  
The VBF non-resonant $HH$ signal production was generated at LO using the
\textsc{MadGraph}5\_aMC@NLO v2.7.3~\cite{Alwall:2014hca} generator with the 
\textsc{NNPDF3.0NLO}~\cite{Ball:2014uwa} PDF set.
Parton showering and hadronisation were performed using \textsc{Pythia~8.244} with the
A14 set of tuned parameters and the \textsc{NNPDF2.3LO} PDF set.
The \textsc{EvtGen} v1.7.0 program was used for describing the bottom- and charm-hadron decays.
%%The cross-section of the VBF non-resonant $HH$ production,
%%calculated at next-to-next-to-next-to-leading order (N3LO) in QCD~\cite{Dreyer:2018qbw}
%%in the limit in which there is no partonic exchange between the two protons,
%%is $1.73^{+0.03\%}_{-0.04\%}\text{(scale)}\pm 2.1\%(\text{PDF}+\alpha_\text{S})$ fb
%%at $\sqrt{s}=13$ TeV and $\text{m}_{H}=125$ GeV.
The cross-section of the VBF non-resonant $HH$ production calculated at
next-to-next-to-next-to-leading order (N3LO) in QCD in the limit 
in which there is no partonic exchange between the two protons~\cite{Dreyer:2018qbw} is
$1.73^{+0.03\%}_{-0.04\%}\text{(scale)}\pm 2.1\%(\text{PDF}+\alpha_\text{S})$ fb.
% This process is normalised to the SM VBF HH cross section, 
% $\sigma_{VBF}=\SI{1.726}{fb}$ calculated at N3LO QCD~\cite{Dreyer_2018}, 
% times the $bb\tau\tau$ BR, $\sigma_{VBF} \times BR (bb \tau\tau)  = \SI{1.726}{fb} \times 0.0730562561  =  \SI{0.126095098}{fb}$.  
Other non-resonant $HH$ production modes are not considered as their contributions
to the analysis sensitivity are expected to be negligible.

The BSM resonant $HH$ signal from the ggF production of a heavy spin-0 resonance
and its decay into a pair of SM Higgs bosons, $X \rightarrow HH$, was simulated with the 
\textsc{MadGraph}5\_aMC@NLO v2.6.1 generator using the \textsc{NNPDF2.3LO} PDF
set at LO accuracy in QCD.
%%%in the extended Higgs sector of 2HDM.
The simulated events were interfaced to \textsc{Herwig~7.1.3}~\cite{Bahr:2008pv,Bellm:2015jjp}
to model the parton shower, hadronisation and underlying event, 
using the H7.1-Default tune~\cite{Gieseke:2012ft}
and the \textsc{NNPDF2.3LO} PDF set.
The resonant $HH$ signal was simulated for 19 values of the resonance mass, $m_{X}$,
between 251~GeV and 1.6~TeV.
% The width of the heavy scalar $X$ was fixed to 10~MeV.
% The normalisation of these resonant signal samples in the analysis is set to 
% $\sigma \times BR  = \SI{1}{pb} \times BR (bb \tau \tau)  =  0.073056256 \SI{1}{pb}$. 
A dummy cross section of 1 pb is chosen for these samples 
to make combination with other decay channels easier, 
and to ease scaling of the signal 
for the calculation of the cross section limits 
(more details in Section \ref{sec:DiHiggs:results}).
% (with a cross section of 1 pb the limit on the POI, that is mu,
% is directly giving the limit on the cross section in pb). 
% Cross sections of 1 pb have already been excluded by the 36 \ifb\ HH combination over the full scanned mass range~\cite{HDBS-2018-58}.
%%The ATLAS fast detector simulation (REF), which relies on a parameterized response of the calorimeters,
%%was used to produced these signal samples.
%%% The cross-section of the BSM resonant $HH$ production is calculated at LO.
\paragraph{Non-resonant signal reweighting and combination}
The non-resonant signal is sensitive to the parameter $\kl$ 
as described in section~\ref{sec:Theory:kl} 
(TODO: ref back to theory section about klambda dependence).
% represents the deviation of the Higgs boson self-coupling from the SM expected value. 
% The non-resonant $HH$ signal samples are generated for 
% only a limited number of $\kappa_\lambda$ points due to heavy computational cost. 
For the \ggH non-resonant $HH$ production, MC samples are generated at \kl=1 and 10, 
while for \VBFH production, 
MC samples are generated at \kl=0, 1, 2 and 10. 
A sample combination technique is used to model the signal hypothesis at different \kl values.
For the \ggH di-Higgs production, 
a reweighting method described in ~\cite{ATL-PHYS-PUB-2019-007} 
is used to obtain predictions at different $\kl$ values 
in the range $\kl \in \left[-30, 30 \right] $ 
in increments of 0.2 based on a linear combination of 
% truth level  %% how to define truth level? 
samples at $\kl=0, 1$ and 20. 
The remaining $\kl = 10$ sample is used to validate the method.
%  at truth level. 
For each \kl value, a set of weights $w(m_{HH},\kl)$ is evaluated 
by dividing the binned $m_{HH}$ distribution of the \kl target sample over the SM distribution. 
Therefore, weights for any \kl sample can be obtained and they can be used to reweight the SM sample to any \kl value. 
Then a reweighted distribution at analysis level results from 
applying the weights to the SM sample after reconstruction and the selection steps. 
Given the assumption that the kinematic of the \ggH events and their acceptance depend only on $m_{HH}$ variable, 
using the weights $w(m_{HH},\kl)$ the reweighted sample describes correctly any kinematic distribution of a given target \kl value. 
Good closure is found between the distributions obtained from \kl generated and reweighted, 
as shown in section~\ref{sec:DiHiggs:systemmatics}.
% To evaluate the sensitivity of the analyses to \kl, 
% the SM \ggH sample is processed through the analyses and then it is reweighted to different values of \kl. 
% The \bbyy analysis considers only the signal yield variation 
% as it was checked that \kl variation in the $m_{\gamma\gamma}$ shape has negligible impact on the final limits, 
% while the \bbtautau analysis takes into account 
% the MVA output shape changes due to the modification of \kl in addition to the signal yield variation. 

For the \VBFH di-Higgs production, the above reweighting procedure is not valid, 
because the kinematic of the events can not be defined just using a single variable such as $m_{HH}$. 
Therefore three MC samples with $\kl = 1,2,10$ are used, 
the events distributions and the multivariate algorithm output 
(more details in section~\ref{sec:DiHiggs:MVA}) 
for any \kl value are obtained from the linear combination 
of the events distribution and the multivariate output 
of the three samples at analysis level. 
The linear coefficients for combining the three samples defined are defined by the formula 
(which is also valid to evaluate the VBF cross-section as a function of \kl where all the other couplings are set to SM value):
%  \begin{equation}
%     \label{eqn:vbf-linear-comb}
%     % \begin{split}
%         \sigma (\kl) =\left(\frac{\kl^{2}}{9} - \frac{4\kl}{3} + \frac{20}{9} \right) \times \sigma (1) + \left(-\frac{\kl^{2}}{8} + \frac{11\kl}{8} - \frac{5}{4} \right) \times \sigma (2) + \left(\frac{\kl^{2}}{72} - \frac{\kl}{24} + \frac{1}{36} \right) \times \sigma (10) 
%     % \end{split}
%     \end{equation} 
% %  More discussion on $\kappa_\lambda$ parameterisation is available in Appendix \ref{app:kl_reweighting}.

\begin{eqnarray*}
    \sigma (\kl) =\left(\frac{\kl^{2}}{9} - \frac{4\kl}{3} + \frac{20}{9} \right) \times \sigma (1) + \left(-\frac{\kl^{2}}{8} + \frac{11\kl}{8} - \frac{5}{4} \right) \times \sigma (2)
\end{eqnarray*}
% \begin{eqnarray*}
    
% \end{eqnarray*}
\begin{eqnarray}
      + \left(\frac{\kl^{2}}{72} - \frac{\kl}{24} + \frac{1}{36} \right) \times \sigma (10) 
    \label{eqn:vbf-linear-comb}
\end{eqnarray}
%  ggF and VBF kl production
%  The non-resonant di-Higgs production via ggF process was simulated with full next-to-leading order (NLO) corrections and finite top mass effect for $\kappa_\lambda=1$ and 10 using the \POWHEGBOXV{v2} generator~\cite{Nason:2004rx,Frixione:2007vw, Alioli:2010xd, Campbell:2014kua} and the PDF4LHC15\_nlo\_30\_pdfas~\cite{Butterworth:2015oua} parton distribution function (PDF) set.
%  Parton showers and hadronization were simulated with \PYTHIAV{8.244}~\cite{Sjostrand:2014zea} using NNPDF23LO with the A14 tune. In addition, in order to evaluate the systematic uncertainties related to parton showers, alternative samples showered with \HERWIGV{v7.1.03}, using the H7.1-Default tune for underlying events and the H7-MMHT2014LO (68cl) tune for parton shower and hadronization, are also generated for the SM and $\kappa_\lambda=10$ case.
 
%  The non-resonant di-Higgs production via the VBF process was simulated at leading-order (LO) for $\kappa_\lambda=0, 1, 2$ and 10 using \MGMCatNLOV{2.6.0}~\cite{Alwall:2014hca} with the NNPDF~2.3~LO~\cite{Ball:2014uwa} PDF set. 
%  % The VBF samples have been produced for the various coupling values shown in Table~\ref{tab:vbf-coupling-samples}.
%  The values of $\kappa_\lambda, \kappa_{2V}$, and $\kappa_V$ are all 1 in the SM.
%  Parton showers and hadronization were simulated with \PYTHIAV{8.244} using NNPDF23LO with the A14 tune. Alternative samples showered with \HERWIGV{v7} are also generated for $\kappa_\lambda=1$ and 10 for estimation of systematic uncertainties related to parton showers.
 
 
\subsubsection{Simulated background samples}
The $t\bar{t}$ production and the single top-quark events 
in the $Wt$-, $s$- and $t$-channels
were generated using the \textsc{Powheg-Box v2} generator 
together with the \textsc{NNPDF3.0NLO} PDF set.
The simulated events were interfaced to \textsc{Pythia~8.230} 
for parton shower and hadronisation
using the A14 set of tuned parameters 
together with the \textsc{NNPDF2.3LO} PDF set.
For all these simulated top-quark processes, 
the top-quark spin correlations are preserved.
The top-quark mass was set to 172.5~GeV.
%%%The $t\bar{t}$ production cross-section is corrected to the theory prediction calculated
%%%at next-to-next-leading order and next-to-next-to-leading-logarithm (NNLO+NNLL)~\cite{Czakon:2011xx}.
The $t\bar{t}$ production cross-section is calculated
at next-to-next-to-leading order 
and next-to-next-to-leading-logarithm 
(NNLO+NNLL)~\cite{Czakon:2011xx}.
%%The cross-sections of the single top-quark processes are corrected to the theory prediction
%%calculated at NLO~\cite{stop_sch_Xsec,stop_tch_Xsec,stop_Wt_Xsec}.
The cross-sections of the single top-quark processes are calculated at
NLO~\cite{stop_sch_Xsec,stop_tch_Xsec,Kidonakis:2010ux}.
The $t\bar{t}$-$Wt$ interference is handled using the diagram removal scheme.

%% Production of a $Z$ ($W$)~boson in association with a $t\bar t$ pair was modelled using \textsc{Sherpa}~2.2.1~(2.2.8)~\cite{sherpa}, with the NNPDF3.0NNLO PDF set and a dedicated parton shower tuning developed by the \textsc{Sherpa} authors.

Events containing $W$ or $Z$ bosons produced in association with jets,
diboson ($WW$, $WZ$ and $ZZ$) production processes, 
and $t\bar tZ$ ($t\bar tW$)
processes were simulated with the 
\textsc{Sherpa}~2.2.1~(2.2.8) generator~\cite{Bothmann:2019yzt} 
using the 
NNPDF3.0NNLO~\cite{Ball:2014uwa} PDF set
with dedicated parton shower tuning developed by the \textsc{Sherpa} authors.
For simulating the $W$/$Z+$jets events 
the matrix elements were calculated
for up to two partons at NLO and four partons at LO using the 
\textsc{OpenLoops}~\cite{Cascioli:2011va}
and \textsc{Comix}~\cite{Gleisberg:2008fv} matrix-element generators.
The number of expected $W$/$Z+$jets events 
is normalised to the NNLO cross-sections~\cite{Catani:2009sm}.
The diboson production was simulated 
up to one additional parton at NLO and up to three additional
partons at LO using the 
\textsc{OpenLoops} and \textsc{Comix} programs.
The cross-sections from \textsc{Sherpa} at NLO 
are used to normalise the diboson
and the $t\bar tW/Z$ events.
%%Events containing $W$ or $Z$ bosons with jets were simulated with
%%\textsc{Sherpa} 2.2.1 generator~\cite{Bothmann:2019yzt} using the NNPDF3.0NNLO PDF set
%%with dedicated parton shower tuning developed by the \textsc{Sherpa} authors.
%%The number of expected $W$/$Z+$jets events is rescaled using the NNLO cross-sections~\cite{Catani:2009sm}.

%%Diboson $WW$, $WZ$ and $ZZ$ processes were generated using \textsc{Sherpa} 2.2.1 generator,
%%which calculates up to one additional parton at NLO and up to three additional partons at LO.
%%NNPDF3.0NNLO PDF set was used together with dedicated parton shower in \textsc{Sherpa}.
%%The cross-sections from \textsc{Sherpa} at NLO are used to normalize the events.

The SM single Higgs boson production 
is considered as part of
the background in this search, 
and its production modes were simulated using the
\textsc{Powheg-Box~v2} generator and the 
\textsc{NNPDF3.0NLO} PDF set.
%%The simulated events of the single Higgs boson production, in various processes,
%%were generated using \textsc{Powheg-Box v2} generator and NNPDF3.0NNLO PDF set.
%The single Higgs boson production from the gluon--gluon fusion process and vector-boson-fusion process,
%with the Higgs boson decaying into a pair of $\tau$ leptons were also simulated using
%\textsc{Powheg-Box v2} generator and NNPDF3.0NNLO PDF set.
The single Higgs boson production via ggF was simulated at 
NNLO accuracy in QCD using the
Powheg NNLOPS program~\cite{Hamilton:2013fea,Hamilton:2015nsa},
whereas the VBF single Higgs boson production was simulated at 
NLO accuracy in QCD~\cite{Nason:2009ai}.
Both productions were interfaced to \textsc{Pythia~8.212} 
for parton shower and hadronisation
using the 
AZNLO tune~\cite{AZNLOtune} together with the 
\textsc{CTEQ6L1} PDF set~\cite{CTEQ6L1}.
%%%
%%%The cross-section of ggF production of single Higgs boson is done at
%%%N3LO accuracy in QCD and NLO in electroweak (EW) calculations~\cite{deFlorian:2016spz},
The cross-section of ggF production of single Higgs boson 
is based on a computation with
N3LO accuracy in QCD, and 
NLO in the electroweak (EW)
expansion~\cite{deFlorian:2016spz,Anastasiou:2016cez,Anastasiou:2015ema,Dulat:2018rbf,Actis:2008ug},
whereas the 
VBF single Higgs boson production cross-section 
is set to the NNLO(QCD)+NLO(EW)
calculation~\cite{deFlorian:2016spz,Ciccolini:2007jr,Ciccolini:2007ec,Bolzoni:2010xr}.
%%%
%Simulated events for $qq \rightarrow WH$, $qq \rightarrow ZH$ and $gg \rightarrow ZH$
%were generated using \textsc{Powheg-Box v2} generator and NNPDF3.0NNLO PDF set.
%The events were interfaced to \textsc{Pythia 8.212} for parton shower
%and hadronisation using the AZNLO tune~\cite{AZNLOtune} together with the
%\textsc{CTEQ6L1} PDF set~\cite{CTEQ6L1}.
The $qq \rightarrow WH$, $qq \rightarrow ZH$ and $gg \rightarrow ZH$ simulated events
were interfaced to \textsc{Pythia~8.212} for parton shower
and hadronisation using the AZNLO tune together with the
\textsc{CTEQ6L1} PDF set.
The cross-sections are set to the NNLO(QCD)+NLO(EW) calculations for
$qq \rightarrow WH$ and 
$qq \rightarrow ZH$~\cite{Ciccolini:2003jy,Brein:2003wg,Ferrera:2011bk,Brein:2011vx,Ferrera:2013yga,Ferrera:2014lca,Campbell:2016jau},
and to the next-to-leading order and next-to-leading-logarithm (NLO+NLL) in QCD for
$gg \rightarrow ZH$~\cite{Altenkamp:2012sx,Hespel:2015zea,ggzhnll,Harlander:2013mla,Brein:2012ne}.
%%%
For the Higgs boson production in association with a pair of top-quarks 
($t\bar{t}H$),
the simulated events were interfaced to 
\textsc{Pythia~8.230} for parton shower
and hadronisation using the 
A14 set of tuned parameters, and 
\textsc{NNPDF2.3LO} PDF set.
The cross-section of $t\bar{t}H$ production is set to 
NLO calculations~\cite{deFlorian:2016spz}.
The SM single Higgs boson production plays a more important role as a background
to the non-resonant $HH$ search than the resonant $HH$ search due to more
similar kinematics between the SM single Higgs boson production and
the non-resonant $HH$ production.